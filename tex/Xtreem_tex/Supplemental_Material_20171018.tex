\documentclass[11pt,a4paper]{article}
\usepackage{graphicx,psfig,fancyhdr,natbib,subfigure}
\usepackage{epsfig, psfig, epsf}
\usepackage{amsmath, cancel}
\usepackage{amssymb}
%\usepackage{lscape}
\usepackage{dcolumn}% Align table columns on decimal point
\usepackage{bm}% bold math
\usepackage{hyperref,ifthen}
\usepackage{verbatim}



%%%%%%%%%%%%%%%%%%%%%%%%%%%%%%%%%%%%%%%%%%%
%       define Journal abbreviations      %
%%%%%%%%%%%%%%%%%%%%%%%%%%%%%%%%%%%%%%%%%%%
\def\nat{Nat} \def\apjl{ApJ~Lett.} \def\apj{ApJ}
\def\apjs{ApJS} \def\aj{AJ} \def\mnras{MNRAS}
\def\prd{Phys.~Rev.~D} \def\prl{Phys.~Rev.~Lett.}
\def\plb{Phys.~Lett.~B} \def\jhep{JHEP}
\def\npbps{NUC.~Phys.~B~Proc.~Suppl.} \def\prep{Phys.~Rep.}
\def\pasp{PASP} \def\aap{Astron.~\&~Astrophys.} \def\araa{ARA\&A}
\def\jcap{\ref@jnl{J. Cosmology Astropart. Phys.}} 
\def\nar{New~A.R.} 

\newcommand{\preep}[1]{{\tt #1} }

%%%%%%%%%%%%%%%%%%%%%%%%%%%%%%%%%%%%%%%%%%%%%%%%%%%%%
%              define symbols                       %
%%%%%%%%%%%%%%%%%%%%%%%%%%%%%%%%%%%%%%%%%%%%%%%%%%%%%
\def \Mpc {~{\rm Mpc} }
\def \Om {\Omega_0}
\def \Omb {\Omega_{\rm b}}
\def \Omcdm {\Omega_{\rm CDM}}
\def \Omlam {\Omega_{\Lambda}}
\def \Omm {\Omega_{\rm m}}
\def \ho {H_0}
\def \qo {q_0}
\def \lo {\lambda_0}
\def \kms {{\rm ~km~s}^{-1}}
\def \kmsmpc {{\rm ~km~s}^{-1}~{\rm Mpc}^{-1}}
\def \hmpc{~\;h^{-1}~{\rm Mpc}} 
\def \hkpc{\;h^{-1}{\rm kpc}} 
\def \hmpcb{h^{-1}{\rm Mpc}}
\def \dif {{\rm d}}
\def \mlim {m_{\rm l}}
\def \bj {b_{\rm J}}
\def \mb {M_{\rm b_{\rm J}}}
\def \mg {M_{\rm g}}
\def \mi {M_{\rm i}}
\def \qso {_{\rm QSO}}
\def \lrg {_{\rm LRG}}
\def \gal {_{\rm gal}}
\def \xibar {\bar{\xi}}
\def \xis{\xi(s)}
\def \xisp{\xi(\sigma, \pi)}
\def \Xisig{\Xi(\sigma)}
\def \xir{\xi(r)}
\def \max {_{\rm max}}
\def \gsim { \lower .75ex \hbox{$\sim$} \llap{\raise .27ex \hbox{$>$}} }
\def \lsim { \lower .75ex \hbox{$\sim$} \llap{\raise .27ex \hbox{$<$}} }
\def \deg {^{\circ}}
%\def \sqdeg {\rm deg^{-2}}
\def \deltac {\delta_{\rm c}}
\def \mmin {M_{\rm min}}
\def \mbh  {M_{\rm BH}}
\def \mdh  {M_{\rm DH}}
\def \msun {M_{\odot}}
\def \z {_{\rm z}}
\def \edd {_{\rm Edd}}
\def \lin {_{\rm lin}}
\def \nonlin {_{\rm non-lin}}
\def \wrms {\langle w_{\rm z}^2\rangle^{1/2}}
\def \dc {\delta_{\rm c}}
\def \wp {w_{p}(\sigma)}
\def \PwrSp {\mathcal{P}(k)}
\def \DelSq {$\Delta^{2}(k)$}
\def \WMAP {{\it WMAP \,}}
\def \cobe {{\it COBE }}
\def \COBE {{\it COBE \;}}
\def \HST  {{\it HST \,\,}}
\def \Spitzer  {{\it Spitzer \,}}
\def \ATLAS {VST-AA$\Omega$ {\it ATLAS} }
\def \BEST   {{\tt best} }
\def \TARGET {{\tt target} }
\def \TQSO   {{\tt TARGET\_QSO}}
\def \HIZ    {{\tt TARGET\_HIZ}}
\def \FIRST  {{\tt TARGET\_FIRST}}
\def \zc {z_{\rm c}}
\def \zcz {z_{\rm c,0}}


\newcommand{\sqdeg}{deg$^{-2}$}
\newcommand{\lya}{Ly$\alpha$\ }
%\newcommand{\lya}{Ly\,$\alpha$\ }
\newcommand{\lyaf}{Ly\,$\alpha$\ forest}
%\newcommand{\eg}{e.g.~}
%\newcommand{\etal}{et~al.~}
\newcommand{\cii}{C\,{\sc ii}\ }
\newcommand{\ciii}{C\,{\sc iii}]\ }
\newcommand{\civ}{C\,{\sc iv}\ }
\newcommand{\SiIV}{Si\,{\sc iv}\ }
\newcommand{\mgii}{Mg\,{\sc ii}\ }
\newcommand{\feii}{Fe\,{\sc ii}\ }
\newcommand{\feiii}{Fe\,{\sc iii}\ }
\newcommand{\caii}{Ca\,{\sc ii}\ }
\newcommand{\halpha}{H\,$\alpha$\ }
\newcommand{\hbeta}{H\,$\beta$\ }
\newcommand{\oi}{[O\,{\sc i}]\ }
\newcommand{\oii}{[O\,{\sc ii}]\ }
\newcommand{\oiii}{[O\,{\sc iii}]\ }
\newcommand{\heii}{[He\,{\sc ii}]\ }
\newcommand{\nii}{N\,{\sc ii}\ }
\newcommand{\nv}{N\,{\sc v}\ }

%% From:: /cos_pc19a_npr/LaTeX/proposals/JWST/JWST_ERS/Proposal/lines.tex
%%  
\newcommand{\imw}{$i$--$W3$}
\newcommand{\imwf}{$i$--$W4$}
\newcommand{\rmwf}{$r$--$W4$}
\newcommand{\imwt}{$i$--$W2$}
\newcommand{\wtmwf}{$W3$--$W4$}
%\newcommand{\kms}{km s$^{-1}$}
\newcommand{\cmN}{cm$^{-2}$}
\newcommand{\cmn}{cm$^{-3}$}
%\newcommand{\msun}{M$_{\odot}$}
\newcommand{\lsun}{L$_{\odot}$}
\newcommand{\lam}{$\lambda$}
\newcommand{\mum}{$\mu$m}
\newcommand{\ebv}{$E(B$$-$$V)$}
%\newcommand{\heii}{\mbox{He\,{\sc ii}}}
\newcommand{\cv}{\mbox{C\,{\sc v}}}
%\newcommand{\civ}{\mbox{C\,{\sc iv}}}
%\newcommand{\ciii}{\mbox{C\,{\sc iii}}}
%\newcommand{\cii}{\mbox{C\,{\sc ii}}}
%\newcommand{\nv}{\mbox{N\,{\sc v}}}
\newcommand{\niv}{\mbox{N\,{\sc iv}}}
\newcommand{\niii}{\mbox{N\,{\sc iii}}}
%\newcommand{\oi}{\mbox{O\,{\sc i}}}
%\newcommand{\oii}{\mbox{O\,{\sc ii}}}
%\newcommand{\oiii}{\mbox{[O\,{\sc iii}]}}
\newcommand{\oiv}{\mbox{O\,{\sc iv}}}
\newcommand{\ov}{\mbox{O\,{\sc v}}}
\newcommand{\ovi}{\mbox{O\,{\sc vi}}}
\newcommand{\ovii}{\mbox{O\,{\sc vii}}}

%\newcommand{\feii}{\mbox{Fe\,{\sc ii}}}
%\newcommand{\feiii}{\mbox{Fe\,{\sc iii}}}
%\newcommand{\mgii}{\mbox{Mg\,{\sc ii}}}
\newcommand{\neii}{[Ne\,{\sc ii}]\ }
\newcommand{\neiii}{[Ne\,{\sc ii}]\ }
\newcommand{\nev}{Ne\,{\sc v}\ }
\newcommand{\nevi}{[Ne\,{\sc vi}]\ }
\newcommand{\neviii}{\mbox{Ne\,{\sc viii}}}
\newcommand{\aliii}{\mbox{Al\,{\sc iii}}}
\newcommand{\siii}{\mbox{Si\,{\sc ii}}}
\newcommand{\siiii}{\mbox{Si\,{\sc iii}}}
\newcommand{\siiv}{\mbox{Si\,{\sc iv}}}
%\newcommand{\lya}{\mbox{Ly$\alpha$}}
%\newcommand{\lyb}{\mbox{Ly$\beta$}}
\newcommand{\hi}{\mbox{H\,{\sc i}}}
\newcommand{\snine}{\mbox{[S\,{\sc ix}]}}
\newcommand{\sivi}{\mbox{[Si\,{\sc vi}]}}
\newcommand{\sivii}{\mbox[{Si\,{\sc vii}]}}
\newcommand{\siix}{\mbox{[Si\,{\sc ix}]}}
\newcommand{\six}{\mbox{[Si\,{\sc x}]}}
\newcommand{\sixi}{\mbox{[Si\,{\sc xi}]}}
\newcommand{\caviii}{\mbox{[Ca\,{\sc viii}]}}
\newcommand{\arii}{\mbox{[Ar\,{\sc ii}]}}

%%[Ar II] 6.97
%% [S IX] 1.252 μm 328 
% [Si X] 1.430 μm 351 
% [Si XI] 1.932 μm 401 
% [Si VI] 1.962 μm 167 
% [Ca VIII] 2.321 μm 128 
% [Si VII] 2.483 μm 205 
% [Si IX] 3.935 μm 303
% [Ar II] 6.97


%\snine\ at 1.252$\mu$m, \six\ at 1.430$\mu$m, \sixi\ at 1.932$\mu$m, \sivi\ at
%1.962$\mu$m, \caviii\ at 2.321$\mu$m, \sivi\ at 2.483$\mu$m \siix\ at
%3.935$\mu$m and \arii\ at 6.97$\mu$m. 
%%
%% such as [Ne ii]12.8 μm, [Ne v]14.3 μm, [Ne iii]15.5 μm, [S iii]18.7 μm and 33.48 μm, [O iv]25.89 μm and [Si ii]34.8 μm (e.g
%%
%% MIR emission lines like [NeII] and [NeV] are ..
%%
%% Also,  arXiv:astro-ph/0003457v1 
%% [NeV] 14.32um & 24.32um and [NeVI] 7.65um imply an A(V)>160 towards the NLR...
%% [NeIII]15.56um/[NeII]12.81um
%%
%% [Ne V] 14.3, 24.2 μm 97.
%% [Ne II] 12.8 μm
%% [OIV] 26μm
%%


\begin{document}

   \title{Supplemental Material for J110057}
\maketitle


\section*{Selection in SDSS-III BOSS of J110057}
SDSS J110057.71-005304.5 was discovered by the Sloan Digital Sky
Survey in 2000.  J110057 was imaged in Run 756 and satisfied a number
of spectroscopic targeting flags\footnote{SERENDIP\_BLUE, ROSAT\_D,
ROSAT\_C, ROSAT\_B, QSO\_SKIRT, ROSAT\_A, see \citet{EDR} and
\citet{Richards2002} for flag descriptions.}  making it an SDSS quasar
target. A spectrum was obtain on MJD 51908, on Plate 277, Fiber 212,
and the spectrum of a $z=0.378\pm0.00003$ quasar was catalogued 
in the SDSS Early Data Release \citep{Schneider2002}.

The second epoch spectrum is from the SDSS-III Baryon Oscillation
Spectroscopic Survey \citep[BOSS; ][]{Dawson2013} and shows the dramatic
downturn at $\lesssim$4300\AA\ .  SDSS-III BOSS actively vetoed {\it
against} low-$z$ QSOs \citep{Ross2012}, and it was due to J110057 being
selected as an ancillary target via a White Dwarf program
\citep{Kepler2015, Kepler2016} that a second spectral epoch was serendipitously
obtained.  Since J110057 was not a BOSS QSO target, it is not subject
to the ``blue offset'' present for BOSS QSO targets \citep{Margala2016}.

A third epoch spectrum was obtained from the Palomar 5m telescope
using the DBSP instrument.  Two exposures of 600s+300s were taken in
good conditions.  A features to note includes the continuum straddling
MgII being blue in the 2017 spectrum, as it was for the SDSS spectrum
in 2000, as opposed to red, as it was for the BOSS spectrum in 2010.

J110057 is in the DECaLS DR3, where there are 8, 3 and 9 exposures in
the $g-$, $r-$ and $z-$band respectively. The $g-$ and $r-$band
observations have fairly limited time spans, ($56707 \leq g_{\rm MJD}
\leq 56727$ and $56367 \leq r_{\rm MJD} \leq 56367$) and are separated
by roughly a year. The $z$-band observations span almost 3 years
($56383 \leq z_{\rm MJD} \leq 57398)$.  The DESI imaging brick name is
1651m010.


\section*{Selection in NEOWISE-R of J110057}
%%  This is text from Aaron's email from 15th Feb, 2017, sent to Daniel, 
%%  for the WISE J1052+1519 outline/paper

W1 and W2 lightcurves for $\sim$200,000 SDSS spectroscopic quasars
were extracted from Data Release 3 (DR3) of the Dark Energy Camera
Legacy Survey (DECaLS). These light curves span from the beginning of
the WISE mission (2010 January) through the first-year of NEOWISER
operations (2014 December). In detail, the W1/W2 light curves are
obtained by performing forced photometry at the locations of
DECam-detected optical sources. This forced photometry is performed on
time-resolved unWISE coadds, each of which represents a stack with
depth of coverage $\sim$12 exposures. A given sky location is observed
by WISE for $\sim$1 day once every six months, which means that our
forced photometry light curves typically have four coadd epochs
available. Coadd epochs of a given object are separated by a minimum
of six months and a maximum of four years. Our coaddition removes the
possibility of probing variability on $\lesssim$1 day time scales, but
it allows us to push $\sim$1.4 magnitudes deeper than individual
exposures while removing virtually all single-exposure artifacts
(e.g. cosmic rays and satellites).

$\sim$30,000 of the SDSS quasars with such W1/W2 lightcurves available
are ``IR-bright'', in the sense that they are above both the W1 and W2
single exposure thresholds and therefore detected at very high
significance in our coadds. For this ensemble of objects, the typical
variation in each quasar's measured (W1-W2) color is 0.06 magnitudes,
which includes statistical and systematic errors expected to
contribute variations at the few hundredths of a mag level. The
typical measured single-band scatter is 0.07 magnitudes in each of W1
and W2. A full characterization of the typical mid-IR quasar
variability will be presented separately (Ross et al., in
preparation).

We undertook a search for extreme outliers relative to these trends.
Specifically, we selected objects with the following characteristics.

\begin{itemize}
\item Monotonic variation in both W1 and W2.
\item W1 versus W2 \textit{flux} correlation coefficient $>$0.9.
\item $>0.5$ mag peak-to-peak variation in either W1 or W2.
\end{itemize}

This yielded a sample of 248 sources. 31 of these are assumed to be blazars 
due to the presence of FIRST radio counterparts. Another 22 are outside of the 
FIRST  footprint, leaving 195 quasars in our IR-variable sample. 

% We randomly selected five of these objects for follow-up spectroscopy
% with Palomar DBSP on the night of 30 January 2017. WISE J1052+1519,
% one of these five, faded by 0.75 (0.9) mags in W1 (W2), and thus
% became 0.15 mags bluer in (W1-W2), making it a significant outlier in
% both single-band and IR color variability.

A link to the key properties of our sample can be found here:
\href{http://portal.nersc.gov/project/cosmo/temp/ameisner/qso\_pages\_v01/}
{\tt qso\_pages\_v01} and the links to the catalogs are given here:
\href{http://portal.nersc.gov/project/cosmo/temp/ameisner/dr3_wise_lc_sample.fits.gz}{{\tt
dr3\_wise\_lc\_sample.fits.gz}} and here:
\href{http://portal.nersc.gov/project/cosmo/temp/ameisner/dr3_wise_lc_metrics_all_qso.fits.gz}{{\tt
dr3\_wise\_lc\_sample.fits.gz}}.  The first catalog has 248 rows,
which are the selected highly IR-variable sample of objects.  The
second catalog is the full \hbox{200 622} quasar sample quasars that
have ``good'' WISE light curves available in DECaLS DR3. In each file,
there are 3 extensions: ex = 0 -- WISE light curve summary metrics; ex
= 1 -- DECaLS DR3 Tractor data for each object; ex = 2 -- SDSS data
for each object.



\section*{Relevant Timescales}
%— Nic, Saavik, Barry to begin to sketch-up a “cartoon” to show the
%time evolution of the a model and system here. This sound look to
%include descriptions of the the e.g. sound speed, disk height and
%pressure support as the cooling front propagates out and the heating
%front gets established and propagates inward.

Our model consists of three main parts: {\it (i)} an event that causes
an initial change of torques at the ISCO; {\it (ii)} the propagation
of an cooling front in the accretion disk outward from the central
engine and an associated increase in the number density of scatterers
and {\it (iii)} a propagation in towards the central black hole of a
heating front (leading to a 'recovery' in the observed spectra).

For out data and observations of J11057, we have to remain agnostic to
the events that initially causes the initial change of torques at the
ISCO.  A change in magnetic field being the cause of non-zero torques
is well studied, see e.g., \citet{Krolik1999, Gammie1999,
Agol_Krolik2000, Reynolds_Armitage2001}.  Notably,
\citet{Afshordi_Paczynski2003} argue that a thick disk at the ISCO
should produce a substantial torque, while a thin one will not.
Regardless of the initial trigger, we suggest some event causes 
a change in the torque conditions at the ISCO. As such, we build 
on the non-zero torque model from \citet{Zimmerman2005}. 

{\bf NPR NOTE:: Also:: \citet{Gardner_Done2017} 
(see their figure 7 cartoon) for obs evidence of interior puffy structure
and \citet{Jiang_Stone_Davis2016}. 
}

For the second part and stage of our model, we need to explain the 
observed light-curves and optical spectra. 
%
J110057 changes from $\approx$17.9mag to $\approx$18.5mag in the
observed $g$-band over $\lesssim$100 days (in the observed frame).
This translates to a Johnson V-band flux density change from 0.262 mJy
to 0.151 mJy.  However, given the redshift of $z=0.378$, the observed
V-band 380-750nm corresponds to 276-544nm in the rest frame ($-6.56 <
\log (\lambda) < -6.26$) or near-UV to red in the quasar frame. The
source flux density drops to 58\% of the original flux in $\sim$3
months. %How does this compare with some of the modelling in the
previous section?

Simply changing the boundary condition at $R_{\rm in}$ from non-zero
torque to zero torque (e.g. collapsing a puffed-up disk inner edge,
or, as mentioned above, the magnetic fields) leads to the difference
in the SEDs ({\bf NPR NOTE:: Figures 6 and 7 in our notes}.  At $\log
\lambda = −6.56$, the flux for $R_{in} = 9 r_{g}$ (dark blue in both)
drops by $\sim0.2$ dex from $\sim38.0$ to $\sim37.8 $ or from $10.0$
to 6$\times 10^{37}$ ergs), or to 60\% of the initial flux density,
consistent with the values above.

However, in the restframe spectrum, the 300 nm flux seems to drop by a
factor of $\sim$5 (given the uncertainty in the normalization). The
flux at $\lambda < $350 nm is dropping relative to the optical
flux. In order to make the multi-color blackbody spectrum do this, we
would need dim large regions of the inner disk simultaneously.  For
example, if the entire inner disk at $R \leq 50 r_{g}$ changed state
and became dimmer on thermal timescales at each annulus, we can
reproduce both the change at short wavelengths and the observed V-band
change.
We can parameterize the relevant disk timescales at $R\sim 50 r_{g}$ as:


There are several relevant timescales associated with J110057 and our 
model explaining the observations. These are: 

``Relevant disk timescales around a $M_{\rm BH} = 10^{8} M_{\odot}$ at $R\sim 25r_{g}$ are:''
\begin{table}
  \centering
  \begin{tabular}{ l c c }
\hline \hline
    Description           &                             & quantity \\
\hline  
&& \\
Orbital                      & $t_{\rm orb}$          &  $6\; {\rm days}
                                                                     \left( \frac{R}{50 r_{g}}  \right)^{3/2}
                                                                     \frac{ r_{g}}{c}$\\
&& \\
Thermal                    &  $t_{\rm thermal}$    &  $6 \; {\rm months} 
                                                                        \left( \frac{\alpha}{0.03}  \right)^{-1}
                                                                        \left( \frac{R}{50 r_{g}}  \right)^{3/2}
                                                                         \frac{ r_{g}}{c}  $ \\
&& \\
Front propagation   & $t_{\rm front}$             & $11 {\rm yr}
                                                                         \left( \frac{h/R}{0.05}  \right)^{-1}
                                                                         \left( \frac{\alpha}{0.03}  \right)^{-1}
                                                                        \left( \frac{R}{50 r_{g}}  \right)^{3/2}
                                                                      \frac{ r_{g}}{c}$\\
&& \\
 Cloud-crushing      &  $t_{\rm cc}$             &   $100\; {\rm yr} \; 
                                                                       \left(  \frac{  \rho_{\rm cloud}/\rho_{\rm medium}}{10^6}  \right )^{1/2} 
                                                                       \left( \frac{R}{4\times10^{10} \rm{km}} \right)
                                                                     \left(\frac{V_{\rm rel}}{10^4 {\rm \; km/s }} \right)^{-1} $\\
&& \\
viscous                     & $t_{\rm \nu}$             & $230 {\rm yr}
                                                                         \left( \frac{h/R}{0.05}  \right)^{-2}
                                                                         \left( \frac{\alpha}{0.03}  \right)^{-1}
                                                                        \left( \frac{R}{50 r_{g}}  \right)^{3/2}
                                                                      \frac{ r_{g}}{c}$\\
&& \\
\hline \hline
    \end{tabular}
    \centering
  \end{table}
where 
$\alpha$ is the viscosity of the disk; 
$\rho_{\rm cloud}$ is ;
$\rho_{\rm medium}$ is  ;
$R$ is the radius of the accretion disk; 
$r_{g}$ is the Gravitational radius; 
$V_{\rm rel}$ is

``Relevant disk timescales around a $M_{\rm BH} = 10^{8} M_{\odot}$ at $R\sim 25r_{g}$ are:''

\section{Comparison with Guo et al. 2016}
Figure~1 of Guo et al. (2016) shows a UV collapse (green curve) in
their source (J23+) at a very similar wavelength to that in the 2010
spectrum of our source (J110057), and with turnover in both around
$\sim$350 nm.  The collapse in J23 happens in 23 days \citep[Figure 2
of ][]{Guo2016} and this object was observed with SDSS back on MJD
52177 (normal) and then 23 days later on MJD 52200. There is a drop of
1.2mag in the $u$-band and 0.5mag in $g$-band, with $r,i,z$ are all
consistent with the earlier observation. J23 is observed with SDSS
again about a year later and $(u,g,r,i,z)$ are all back consistent
with `normal'. More importantly, XMM-Newton spectra straddle this time
period. The XMM-Newton spectra are from June 3 and Nov. 28 2001. Both
spectra are consistent with {\it no neutral absorption in the
rest-frame}. So the sightline is clear on both of those dates
(i.e. the Nov. spectrum is 45 days after the UV
catastrophe). \citet{Guo2016} also find that the IR does not
significantly change and that the broad lines are consistent with
being constant over time.

%I don’t think we have any data from CRTS on this source going back to this time (the stuff Matthew sent goes from MJD 53500-now). Correct me if I’m wrong.

\citet{Guo2016} discuss two scenarios to explain this behaviour: 1)
inner accretion disk change and 2) eclipse by an optically thick
cloud. They make the point that in principle both models could
explain the observation. In scenario 1) they suggest that turning off
the disk at $< 60rg = 30 rSch$ would explain this (in the same manner
as we discuss for J11057) but they find this implausible since
``quasars are not observed to flicker like this typically''.
\citet{Guo2016} then spend more time on scenario 2). So, based on the
initial optical spectrum (23 days before the $u$-band dip) and the
Nov. X-ray spectrum (45 days after the $u$-band dip) they say that the
dip lasts $< 65$ days. 



\bibliographystyle{mn2e}
\bibliography{/cos_pc19a_npr/LaTeX/tester_mnras}


\end{document}