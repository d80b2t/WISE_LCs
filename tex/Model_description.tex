



Our model consists of three main stages: {\it (i)} an event that causes
an initial change of torques at the ISCO; {\it (ii)} the propagation
of an cooling front in the accretion disk, outward from the central
engine with an associated increase in the number density of scatterers
and {\it (iii)} a propagation back in towards the central black hole of a
heating front, leading to a `recovery' in the observed spectra.

The current dataset and observations of J11057 means we remain agnostic to
the event(s) that causes the initial change of torques at the
ISCO.  A change in magnetic field being the cause of non-zero torques
is well studied, see e.g., \citet{Krolik1999, Gammie1999,
Agol_Krolik2000, Reynolds_Armitage2001}, but we currently 
cannot confirm this. Notably,
\citet{Afshordi_Paczynski2003} argue that a thick disk at the ISCO
should produce a substantial torque, while a thin one will not.
Regardless of the initial trigger, we suggest some event causes 
a change in the torque conditions at the ISCO. As such, we build 
on the non-zero torque model from \citet{Zimmerman2005}. 

For the second stage of our model, we need to explain the
observed light-curves and optical spectra. J110057 changes from
$\approx$17.9mag to $\approx$18.5mag in the observed $g$-band over
$\lesssim$100 days (in the observed frame).  This translates to a
Johnson V-band flux density change from 0.262 mJy to 0.151 mJy.
However, given the redshift of $z=0.378$, the observed V-band
380-750nm corresponds to 276-544nm in the rest frame ($-6.56 < \log
\lambda < -6.26$) or near-UV to red in the quasar frame. The source
flux density drops to 58\% of the original flux in $\sim$3 months.

Simply changing the boundary condition at $R_{\rm in}$ from non-zero
torque to zero torque (e.g. collapsing a puffed-up disk inner edge,
or, as mentioned above, the magnetic fields) leads to the difference
in the SEDs ({\bf NPR NOTE:: Figures 6 and 7 in the notes}).  At $\log
\lambda = −6.56$, the flux for $R_{in} = 9 r_{g}$ %(dark blue in both)
drops by $\sim0.2$ dex from $\sim38.0$ to $\sim37.8 $ or from $10.0$
to 6$\times 10^{37}$ ergs), or to 60\% of the initial flux density,
consistent with the values above.

In the restframe spectrum, however, the 300 nm flux seems to drop by a
factor of $\sim$5 (given the uncertainty in the normalization). The
flux at $\lambda < 350$ nm is dropping relative to the optical
flux. In order to make the multi-color blackbody spectrum do this, 
large regions of the  inner disk have to dim simultaneously. 
For example, if the entire inner disk at $R \leq 50 r_{g}$ changed state
and became dimmer on thermal timescales at each annulus, we can
reproduce both the change at short wavelengths and the observed V-band
change.

\citet{Gardner_Done2017} present observational evidence (see their
figure 7 cartoon) for an interior puffy structure while
\citet{Jiang_Stone_Davis2016} examine in detail the effects of the
bound-bound transitions in iron and the his iron opacity ``bump'', on
the thermal stability and vertical structure of
radiation-pressure-dominated accretion disks, relevant in UV emitting
regions of the accretion disk flow.
