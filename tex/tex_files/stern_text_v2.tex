\documentclass{mnras}
\begin{document}
\section{Possible explanations and predictions}
\label{sec:explanations}
In this section, we discuss likely explanations for our observations and we make testable predictions for each of the models we outline. 
There are two broad categories of explanation for our observations: obscuration and changes to the inner disk.

\subsection{Obscuration}
First we consider dimming of the central continuum due to obscuration along the sightline. On one hand, if there is no time-lag between the drop in the  optical/UV ('high energy') continuum and the reprocessed IR ('low energy') continuum, then the obscurer must lie at some distance from the reprocessing region. The size of the obscurer and the odds of an eclipse-event obscuring half of the low-energy region drop rapidly the further the obscurer is from the low-energy region. On the other hand, if there is a time-lag between the drop in the high-energy and low-energy continua, an obscuration model implies an obscurer lying between the high energy and low-energy regions.  Based on our observations, such an obscurer needs to hide $\sim 1/2$ of the high-energy continuum from the reprocessing region and $\sim$ all of the high-energy continuum from our sightline. A large-scale change in the nature of the inner disk could do this (see below), but a typical broad-line region cloud ($N_{H} \sim 10^{21-22}\rm{cm}^{-2}$, REF) will not reduce the 2-10keV X-ray continuum by an order of magnitude. We therefore think obscuration independent of inner disk changes is unlikely as a cause of the observed effects. 

\subsection{Changes to the Inner disk}
\label{sec:times}
For possible explanations involving the gas disk, we need estimates of the relevant timescales for processes at small radii. Four timescales are important to consider for our purposes. The orbital timescale in the accretion disk is approximately $t_{\rm orb} \sim 1/\Omega$ where $\Omega=\sqrt{GM_{\rm SMBH}/R^{3}}$ is the Keplerian orbital angular frequency. The thermal timescale in the disk is $t_{\rm th} \sim t_{\rm orb}/\alpha$, where $\alpha$ is the disk viscosity parameter \citep{Shakura73}. Cooling and heating front timescales cross the disk on timescales of $t_{\rm front} \sim (h/R)^{-1} t_{th}$, where $h/R$ is the disk aspect ratio, with $R$ the disk radius and $h=c_{s}/\Omega$ the disk height, with $c_{s}$ the local disk sound speed at radius $R$. The viscous disk timescale is written as $t_{\nu}=(h/r)^{-2}t_{\rm th}$. 

Let us assume a SMBH of mass $M_{\rm SMBH}=3 \times 10^{8}M_{\odot}$ and luminosity $L \sim 0.032L_{\rm Edd}$ (REF), where $L_{\rm Edd}$ is the Eddington luminosity. Then, the characteristic distance scale is $r_{g}=GM_{\rm SMBH}/c^{2} \sim 4.5 \times 10^{11}$m ($\sim 3$AU). Assuming $L =\eta \dot{M}c^{2}$ is the luminosity due to accretion, where $\eta \sim 0.1$ is the luminosity efficiency of accretion, the characteristic mass-flow rate is
\begin{equation}
\dot{M}_{\rm SMBH} \approx 0.2\frac{M_{\odot}}{yr} \left( \frac{\eta}{0.1}\right)^{-1} \left( \frac{M_{\rm SMBH}}{3 \times 10^{8}M_{\odot}}\right)\left(\frac{L_{\rm Edd}}{0.032}\right).
\label{eq:m_dot}
\end{equation}
Assuming a standard (relatively) thin-disk AGN model consisting of multiple annuli at temperatures that drop with radius, where the mass flow rate across each annulus is $\dot{M}_{\rm SMBH}=3\pi\nu\Sigma$ and $\nu$ is the viscosity, the disk temperature drops as $T \propto r^{-3/4}$ (REFs). The effective disk temperature (the observed 'photosphere' or surface of the disk) is at temperature $T_{\rm eff}^{4} \sim T^{4}/\tau$ where $\tau=\kappa \Sigma$ and $\kappa$ is the opacity parameter. In order for the observed high-energy flux to drop dramatically at $\leq 3500\AA$ requires the thin-disk luminosity at $R<150r_{g}$ to drop by $\sim 2$ orders of magnitude. If some small UV flux persists, the thin disk luminosity at $R<150r_{g}$ can drop by an order of magnitude. 

We can parameterize the relevant disk timescales at $R\sim 150r_{g}$ as:
\begin{eqnarray}
t_{\rm orb} & \sim & 1{\rm month} \left(\frac{R}{150r_{g}}\right)^{3/2} \frac{r_{g}}{c}\\
t_{\rm th} & \sim & 3{\rm yr} \left(\frac{\alpha}{0.03}\right)^{-1}\left(\frac{R}{150r_{g}}\right)^{3/2} \frac{r_{g}}{c}\\
t_{\rm front} & \sim & 60{\rm yr} \left(\frac{h/R}{0.05}\right)^{-1}\left(\frac{\alpha}{0.03}\right)^{-1}\left(\frac{R}{150r_{g}}\right)^{3/2} \frac{r_{g}}{c}\\
t_{\nu} & \sim & 1200{\rm yr} \left(\frac{h/R}{0.05}\right)^{-2}\left(\frac{\alpha}{0.03}\right)^{-1}\left(\frac{R}{150r_{g}}\right)^{3/2} \frac{r_{g}}{c}.
\end{eqnarray}
Evidently, since we observe changes on timescales of years in this source, any changes in the disk must be occurring on the thermal or dynamical timescales and are unlikely to be due to viscous effects or propagation of cooling/heating fronts in the inner disk.

\subsection{Causes of changes in inner disk}
\label{sec:disk}
Changes in the innermost disk can occur due to occasional instabilities or local perturbations due to objects or events locally or elsewhere in the disk. 
\subsubsection{Disk Instabilities}
Disk instabilities can occur for a variety of reasons and on a range of timescales \citep[e.g.][]{SS76}, Lightman \& Eardley 1976, Abramowicz et al. 1988). The inner part of AGN accretion disks are expected to be supported mainly by radiation pressure and these regions can be unstable. The instability condition at a given disk radius can be characterized as \citep{SS76,Abram88}
\begin{equation}
\left.\frac{\partial \ln Q^{+}}{\partial \ln P}\right|_{\Sigma}> \left.\frac{\partial \ln Q^{-}}{\partial \ln P}\right|_{\Sigma}
\end{equation}
where $Q^{+}$ is the rate of viscous heating, $Q^{-}$ is the rate of cooling due to vertical radiation flux and horizontal advection, and $P=P_{g}+P_{r}$ is the total pressure due to gas pressure and vertical radiation pressure evaluated at AGN disk radius with a given surface density $\Sigma$. 
Moderately short timescale instabilities can occur in e.g. ($\Sigma,T_{\rm eff}$) or ($\dot{m},\Sigma$) parameter space due to modest changes in $\dot{m}$ or $\Sigma$ (e.g. Lin \& Shields 1986, Abramowicz et al. 1988). If $\dot{\Sigma}$ changes locally so that the local disk lies on the unstable part of the ($\Sigma, T_{\rm eff}$) parameter space S-curve. Once this happens, $T_{\rm eff}$ can drop by an order of magnitude on approximately the thermal timescale $t_{\rm th}$ (or some small multiple thereof, \citet{SS76}). Since the midplane temperature $T$ is approximately unchanged, but $T^{4}=\tau T_{\rm eff}^{4}$, $\tau$ can increase dramatically. Thus, the disk surface must move away from the heat source (mid-plane) and so the disk expands.

\subsubsection{Perturbations due to objects/events}
\label{sec:emri}
Changes in the inner disk state can occur due to the presence of local perturbers, such as an extreme mass ratio inspiral (EMRI) event, or more distant changes in the accretion flow.
A change in the local value of $\dot{\Sigma}$ or $\dot{M}$, which promotes instabilities in ($\Sigma,T_{\rm eff}$) such as those described above, may simply be due to stochastic instabilities in the accretion flow, or might occur due to e.g. embedded supernovae in AGN disks \citep{McK14,McK17}, which will cause an infall of low-angular momentum gas on the timescale of the shockfront expansion through the disk. Stalling of in-migrating objects in the AGN disk might also create a change in $\dot{\Sigma}$ or $\dot{M}$ in the innermost disk \citep{Bello16}. 

A large population of stellar mass black holes, stellar remnants and stars are expected in AGN disks \citep[e.g.][]{Syer91,Arty93,McK12}. Torques from gas in the disk causes these secondary objects to migrate in the disk and a fraction of the secondaries will end up on the central supermassive black holes in an extreme mass ratio inspiral (EMRI) event. Once a secondary object ends up in the innermost regions of the accretion disk, its mass can become comparable to, or even dominate, the co-rotating disk mass. From eqn.~\ref{eq:m_edd}, a stellar mass black holes of $\sim 10M_{\odot}$ could dominate the innermost gas flow on a timescale of decades. Therefore the spectral output of the inner disk can change on the timescales of the EMRI.

The decay timescale of an isolated stellar mass black hole around a supermassive black hole is \citep{Peters64}  
\begin{equation}
t_{GW} \approx \frac{5}{128} \frac{c^{5}}{G^{3}} \frac{a_{b}^{4}}{M_{b}^{2} \mu_{b}} (1-e_{b}^{2})^{7/2}
\label{eq:t_gw}
\end{equation}
where $M_{b}=M_{SMBH}+M_{2}$ is the binary mass, $\mu_{b}=M_{SMBH}M_{2}/M_{b}$ is the binary reduced mass and $(a_{b},e_{b})$ are the initial binary semi-major axis and eccentricity. Re-parameterizing we find
\begin{equation}
t_{GW} \approx 620\rm{yr} \left( \frac{a_{b}}{10r_{g}}\right)^{4} \left( \frac{M_{SMBH}}{10^{7}M_{\odot}}\right) \left(\frac{q}{10^{-6}} \right)^{-1}
\end{equation}
where $q=M_{2}/M_{SMBH}$ is the binary mass ratio.

\subsection{Model Predictions}
Disk instabilities occur on short timescales, but depending on the cause of the instability, we may wait a long or short time for a change of state.
If we are observing a AGN disk undergoing a state change, the number of AGN which undergo such changes in a observing period provides a constraint on the likely limit cycle.

\begin{thebibliography}{99}
\bibitem[\protect\citeauthoryear{Artymowicz et al.}{1993}]{Arty93} Artymowicz P., Lin D.N.C. \& Wampler E.J., 1993, ApJ, 409, 592
\bibitem[\protect\citeauthoryear{Bellovary et al.}{2016}]{Bello16} Bellovary J., Mac Low M.-M., McKernan B. \& Ford K.E.S., 2016, ApJ, 819, L17
\bibitem[\protect\citeauthoryear{McKernan et al.}{2012}]{McK12} McKernan B., Ford K.E.S., Lyra W. \& Perets H.B., 2012, MNRAS, 425, 460
\bibitem[\protect\citeauthoryear{McKernan et al.}{2014a}]{McK14} McKernan B., Ford K.E.S., Kocsis B., Lyra W. \& Winter L.M., 2014, MNRAS, 441, 900
\bibitem[\protect\citeauthoryear{McKernan et al.}{2017}]{McK17} McKernan B., Ford K.E.S. et al. 2017, MNRAS, submitted, arXiv:1702.07818
\bibitem[\protect\citeauthoryear{Peters}{1964}]{Peters64} Peters P.C., 1964, Physical Review, 136, 1224 
\bibitem[\protect\citeauthoryear{Shakura \& Sunyaev}{1973}]{Shakura73} Shakura N.I. \& Sunyaev R.A. 1973, A\&A, 24, 337 
\bibitem[\protect\citeauthoryear{Shakura \& Sunyaev}{1976}]{SS76} Shakura N.I. \& Sunyaev R.A. 1976, MNRAS, 175, 613 
\bibitem[\protect\citeauthoryear{Syer et al.}{1991}]{Syer91} Syer D., Clarke C. \& Rees M.J., 1991, MNRAS, 250, 505 

\end{thebibliography}


\end{document}