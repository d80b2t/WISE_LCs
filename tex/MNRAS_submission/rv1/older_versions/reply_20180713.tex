\documentclass[11pt, a4paper]{article}

\usepackage{lipsum}
\usepackage[top=0.75in, bottom=1.25in, left=0.75in, right=0.75in]{geometry}

\begin{document}

\begin{center}
{\bf Author's Reply: MN-18-1820-MJ, Ross et al.}
\end{center}
\hspace{16pt}

\noindent
We very much thank the Reviewer for (what the author's felt!) was a
quick reply and a sincere reading of our paper. We taken on board the
comments raised, and have updated our manuscript accordingly to
address each one of these. Our changes are in {\bf bold font} in the
text.

\medskip
\medskip
\noindent
1. Too much credit is given to Sirko \& Goodman 2003. That paper was
mainly concerned with the implications of self-gravity for the
spectrum emitted by the outer parts of the disc $r \gg r_{g}$. Little
if any consideration was given to the boundary conditions at the inner
edge of the disc (zero vs. nonzero torque). 

\begin{quote}
This is a good point from the referee and is well noted. We have 
removed the egregious reference in the caption of Figure 2 where 
we incorrectly linked Sirko \& Goodman 2003 to the NZT ISCO model. 
Upon checking the rest of the manuscript, we see no other instance 
where the Sirko \& Goodman 2003 model is over credited. 

%% {\bf Note from NPR::} The {\it only} place I see this being an
%% issue is in the caption of Figure 2, where we reference Sirko \&
%% Goodman 2003 when presenting the the inflated disk with NZT ISCO
%% model.

%% Note from Saavik::
%% (We can add more citations of other models if/where appropriate) 
%% and now note in the text that we are not actually *using* Sirko \& Goodman 
%% except as a guide to a semi-realistic  disk temperature structure. 
%% They're good at 100$r_g$ and out,  which is relevant.
\end{quote}


\noindent
2. Because the present authors are concerned with inner parts of the
disc, they should be more careful in considering relativistic
effects. In particular, the ISCO probably lies at $r < 6GM/c^{2} =
3r_{\rm S}$ because AGN black holes have significant rotation (e.g.,
Reynolds 214, SSRv 183, 277; Capellupo et al. 2016, MN 460, 212). The
specific binding energy at the ISCO is therefore probably larger than
$1 - \sqrt{8/9} \approx 0.057$, so that even if there were no torque
at the ISCO, the ratio of the bolometric luminosity to the
monochromatic luminosity at any point in their spectra (e.g. $\lambda
= 5100$\AA\ ) would be larger than they suppose even in a steadily
accreting thin disc. Naturally, however, the ISCO would not be
expected to change on human timescales.

\begin{quote}
We note the point raised by the referee, and have added this caveat to 
the end of Section 3.2. 

%% Note from Saavik::
%% We do want to re-iterate that we e.g. don't argue for major changes in
%% the spin, and indeed one can fit our continua with higher spin models. 
%% One has to move the suppression radius inwards, but otherwise they
%% look quite similar to the non-spinning case--we don't have leverage on
%% the spin.
\end{quote}


\noindent
3. Although I agree that a thermal or viscous instability is a more
plausible interpretation of their data, the arguments given here
against transient absorption/obscuration are not wholly convincing and
could be improved. For the authors’ values of the black-hole mass
$(7\times10^{8} M_{\odot})$ and Eddington ratio (0.07), the rest-
frame emission at $\lambda \leq 5100$\AA\ ) comes from $r \leq 0.01$pc
($\approx$225$r_{g}$, as they estimate). An obscuring cloud just large
enough to block our view of this region, but in circular orbit at the
sublimation radius (0.4 pc) would cross its own diameter in
approximately six years, which is comparable to the time interval
spanned by the spectroscopic observations ($\approx$7.7 yr). However,
were the ``UV collapse'' caused by absorption, there would be other
implications. The authors should determine what (non-grey) reddening
law would be needed to explain the differences among the spectra, and
whether, for any reasonable gas-to-dust ratio and turbulent
broadening, the absorber would produce transient optical absorption
lines (since the optical emission is obscured by only a factor
$\sim$2); and finally, whether such a small absorber could affect the
infrared flux (probably not).

\begin{quote}
We note the point raised by the referee, and have added a paragraph to
the end of Section 3.1 here to address this point. We note this
`Goldilocks' obscuring cloud scenario might, but the physical scenario
won't work for the Guo et al. (2016) objects due to those very short
timescales.

{\bf Note from NPR::} I'm still a bit confused as to the point about 
``what  (non-grey) reddening law would be needed to explain the 
differences among the spectra''. I have a plot that shows these ratios 
(attached) though I'm not sure if this fully address the issue here. 

%%  We would like to
%%  re-iterate the discussion in Section 3.2.2 where we actually already 
%%  describe the greybofy reddening law required to fit the 2010 spectra . 
\end{quote}


\noindent
4. The bald statement (p. 8, col. 2) ``The 2010 spectrum can not be fit
with...a simple MTB model with an alternate temperature profile''
requires justification. Really? The continua cannot be fit even if
$T_{\rm eff}(r)$ is a completely free function? Why not? Of course no
superposition of local blackbodies will explain the emission lines,
but that is already a known limitation of thin-disc models for
``ordinary'' QSOs.

\begin{quote}
We note this good point from the referee and refer to the dotted black line
in Figure 3, which is a fit designed to {\it try} and fit the
turnover, with an arbitrary normalization. 

%% Saavik's notes:: 
%%I've screwed with T(r)
%%too. Maybe I should plot a single blackbody to make them happy? It's
%%too fat. We could improve the caption to highlight this.
\end{quote}


\noindent
5. The authors are too quick to dismiss thermal instabilty along the
lines of the works by Hameury, Lasota, and colleagues for dwarf novae
and X-ray binaries. It is known that radiation-pressure-dominated
discs are thermally unstable (Lightman \& Eardley 1974; Jiang, Stone,
\& Davis 2013). Whereas the authors consider (and then dismiss in
favour of cold absorbers) simple thermal instability in the “hot”
direction—leading to an RIAF—it could equally go in the opposite
direction, leading to a cold, low-$\dot{M}$, gas-pressure-dominated state,
since the growth rates of linear instabilities are independent of the
sign of their initial perturbations. The thermal timescale
$(\alpha\Omega)^{−1}$ at 225$rg$ is $\sim$4 yr, in the right range to
explain these observations. Admittedly, it is worth asking why all
QSOs do not undergo propagating thermal instabilities of this
sort. But perhaps they do: it would be interesting to compare the
numbers of changing-look QSOs with what might be expected on such a
model. To do this, however, would be beyond the scope of the present
paper.

\begin{quote}
We very much like the suggestion of what the various models, including the
ISCO Cooling Front propagation model we propose here, would mean for 
e.g. detection and observation rates of the CLQs. Indeed, we are exploring 
just this issue in our upcoming `demographic' paper (Graham et al. 2018 in 
advanced prep.). 

We slightly disagree with the referee, that we are too quick to dismiss thermal 
instabilty along the lines of the works by Hameury, as we have noted this already 
in the current text and Discussion [the paragraph starting, ``By 2014, the PanSTARRS green and red fluxes begin recovery. This may
be associated with the inward propagation of a heating instability (Hameury 2009).'']. 
\end{quote}

\noindent
Other minor changes::

\begin{quote}
  \begin{itemize}
  \item Updated Figure 2 with the emission line at $\approx$4718\AA\ observed being [Ne\,{\sc v}]\ and not [Ne\,{\sc iii}]\ .
  \item In the Introduction:: ``... changes in the wider accretion disk, including major structural changes out to $\approx$225$r_{g}$'' to be consistent with the models and discussions later in the paper. 
  \item Caption for Figure 2:: ``The dotted red line shows a modified zero-torque model where the thermal disk emission interior to $40 r_{\rm g}$ is suppressed by a factor of 10.''
  \end{itemize}
\end{quote}

\end{document}

