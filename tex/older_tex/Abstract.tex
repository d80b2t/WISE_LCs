While variability is a distinguishing feature of quasars, the extremes
of this behavior have only recently> been systematically probed by new
generations of wide-area, multi-epoch surveys.  A new class of
``changing-look quasars'' has been identified in which the strong UV
continuum and broad hydrogen emission lines associated with quasars
appear or disappear on timescales of years.  The physical processes
responsible for this behaviour are thought to be due to either
obscuration or changes in the inner disk.  Here we report on three
epochs of spectroscopy of SDSS~J110057.70-005304.5, an extreme
mid-infrared variable quasar at $z = 0.378$ whose UV continuum and
broad hydrogen emission lines have dramatically faded over the past 20
years.  An serendipitous archival spectrum of this quasar from 2010
shows an intermediate phase of the transition during which the flux
below rest-frame $\sim 3400$~\AA\ has collapsed.  This is unique
compared to previously published examples of changing-look quasars,
and can only be explained by dramatic changes in the innermost regions
of the accretion disk. Possible triggers include disk instability, or
changes in the X-ray corona above the BH, each of which lead to
distinct predictions for future behaviour of this source and other
quasars. The optical continuum has been rising again since mid-2016,
leading us to predict a rise in hydrogen emission line flux in the
next few months.  If our prediction is confirmed, the physics of
'changing look' quasars are governed by processes at the innermost
stable circular orbit around the black hole, and the structure of the
innermost disk.