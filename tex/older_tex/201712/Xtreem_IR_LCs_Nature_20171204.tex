%%%%%%%%%%%%%%%%%%%%%%%%%%%%%%%%%%%%%%%%%%%%%%%%%%%%%%%%%%%%%
%%
%%
%%     NPR's  Nature template...  
%%     v1.0.0   Thu Dec  8 19:05:57 EST 2016
%%                                                                         
%%
%%%%%%%%%%%%%%%%%%%%%%%%%%%%%%%%%%%%%%%%%%%%%%%%%%%%%%%%%%%%%
\documentclass{nature}
%\documentclass{natureprintstyle}

\input{format}
\bibliographystyle{naturemag}

\title{A new interpretation of optical and infrared variability in quasars}

\author{Nicholas~P.~Ross$^{1}$ et al.  
%K. E. Saavik Forda$^{2,3,4}$
%Matthew Graham$^{5}$
%Barry McKernan$^{2,3,4}$
%Daniel Stern, 
%Aaron Meisner, 
%Andrew Drake, 
%Arjun Dey, 
%David Schlegel, 
%Roberto Assef, 
%Jun, 
%Peter Eisenhardt, 
%Djorgovsk
}
%%
% Saavik~K.~Ford,$^{7}$, Barry~L.~McKernan$^{7}$,
% Daniel~Stern$^{3}$, Aaron~Meisner$^{2}$, 
% Matthew~Graham$^{4}$, 
% Arjun~Dey$^{5}$ \& David~J.~Schlegel$^{2}$ 
% Andrew~Drake$^{4}$, 
% Roberto~J.~Assef$^{6},
% Mislav~Balokovic$^{6}, 
% Murray~Brightman$^{4}, 
% for the DECaLS Consortium. 

\begin{document}

\maketitle

\begin{affiliations}
  \item Institute for Astronomy, University of Edinburgh, Royal Observatory, Blackford Hill, Edinburgh EH9 3HJ, United Kingdom 
%  \item Department of Science, BMCC, City University of New York, New York, NY 10007, USA
 % \item Department of Astrophysics, American Museum of Natural History, New York, NY 10024, USA
 % \item  Graduate Center, City University of New York, 365 5th Avenue, New York, NY 10016, USA
%XSXS\item Cahill Center for Astronomy and Astrophysics, California Institute of Technology, Mail Code 249/17, 1200 E California Blvd, Pasadena CA 91125, USA
 % \item Department of Astrophysics is located in the Rose Center for Earth and Space, American Museum of Natural History, Central Park West at 79th Street, New York, NY 10024, U.S.A
  %\item Jet Propulsion Laboratory, California Institute of Technology, 4800 Oak Grove Drive, Mail Stop 169-221, Pasadena, CA 91109, USA 
  %\item California Institute of Technology, 1200 East California Boulevard, Pasadena, CA 91125, USA
  %\item Steward Observatory, 933 North Cherry Avenue, Tucson, AZ 85721, U.S.A.
%  \item Universidad Diego Portales, Av Republica 180, Santiago, Regi ́on Metropolitana, Chile}
%  \item Lawrence Berkeley National Laboratory, 1 Cyclotron Road, Berkeley, CA 92420, U.S.A. 
\end{affiliations}


\begin{abstract}
Changing-look quasars are a recently identified class of active galaxies in 
which the strong UV continuum and/or broad optical hydrogen emission
lines associated with unobscured quasars either appear or disappear on
timescales of months to years \cite{LaMassa2015, Runnoe2016,
MacLeod2016, Ruan2016, Yang2017}.  The physical processes responsible for this
behaviour are still debated, but changes in the black hole accretion
rate or accretion disk structure appear more likely than changes in
obscuration \cite{Hutsemekers2017, Sheng2017}.  Here we report on
three epochs of spectroscopy of SDSS J110057.70-005304.5, a quasar at
a redshift of $z=0.378$ whose UV continuum and broad hydrogen
emission lines have dramatically faded over the past $\approx$20
years. The change in this quasar was initially seen in the infrared,
and an archival spectrum from 2010 shows an intermediate phase of the
transition during which the flux below rest-frame 3400\AA\ has
collapsed. This is unique compared to previously published examples of
changing-look quasars, and is best explained by dramatic changes in
the innermost regions of the accretion disk. The optical continuum has
been rising again since mid-2016, leading to a prediction of a rise in
hydrogen emission line flux in the next year. If our model is
confirmed, the physics of changing-look quasars are governed by
processes at the innermost stable circular orbit (ISCO) around the
black hole, and the structure of the innermost disk. The easily
identifiable and monitored changing-look quasars would then provide a
new probe of the strong gravity regime.
\end{abstract}



%%%%%%%%%%%%%%%%%%%%%%%%%%%%%%%%%%%%%%%%%%%%%%%%%%%%%%%%%%%%%
%%%%%%%%%%%%%%%%%%%%%%%%%%%%%%%%%%%%%%%%%%%%%%%%%%%%%%%%%%%%%
%%
%%   SECTION 1  SECTION 1  SECTION 1  SECTION 1  SECTION 1  SECTION 1  
%%   SECTION 1  SECTION 1  SECTION 1  SECTION 1  SECTION 1  SECTION 1  
%%   SECTION 1  SECTION 1  SECTION 1  SECTION 1  SECTION 1  SECTION 1  
%%
%%%%%%%%%%%%%%%%%%%%%%%%%%%%%%%%%%%%%%%%%%%%%%%%%%%%%%%%%%%%%
%%%%%%%%%%%%%%%%%%%%%%%%%%%%%%%%%%%%%%%%%%%%%%%%%%%%%%%%%%%%%
%\section{Introduction}
The changing-look quasar phenomenon, where the dramatic disappearance,
or appearance, of the strong UV continuum and/or the prominent broad
optical emission lines is seen on month-to-year timescales, is now
widely observed \cite{LaMassa2015, MacLeod2016, Runnoe2016, Ruan2016,
Gezari2017, Rumbaugh2017, Yang2017} yet poorly understood. Changes in
obscuration are generally disfavoured due to the timescales currently
observed \cite{Hutsemekers2017, Sheng2017}, and it is clear that
changing-look quasars are a key laboratory for understanding accretion
physics and active galactic nuclei (AGN).

Developed in the early 1970s, the Shakura-Sunyaev $\alpha-$disk model
\cite{SS73} of AGN accretion disk explains the strong, blue continuum
observed from AGN as thermal emission from an optically thick,
geometrically thin accretion disk ($h / r \ll 1 $, where $h$ is the
vertical scale height of the disk and $r$ is the radius from the disk
center).  However, the \cite{SS73} $\alpha$-disk thermal model is also
known to have serious short-comings e.g.  \cite{Antonucci1999,
Koratkar_Blaes1999,Lawrence2012}. For example, AGN seem to be cooler
than they ought to be \cite[e.g., ][]{Lawrence2012} with the spectral
energy distributions (SEDs) of AGN showing a universal near-UV shape,
reaching a maximum energy flux around 1100\AA.  Such a peak suggests a
characteristic turnover temperature of T$\sim$30,000K, whereas for a thermal
model, the characteristic temperature should be roughly
T$\sim$100,000K. Moreover, constraints from microlensing observations
for the size of the optical emission region \cite[e.g.,][]{Pooley2007,
Morgan2010, Morgan2012, Mosquera2011} suggest this region is a factor
of $\sim$4 times larger than the one predicted by the standard
Shakura-Sunyaev disk.

Changing-look quasars have traditionally been discovered by looking
for large, $| \Delta m | >1$ magnitude changes in optical light curves
of quasars or galaxies (e.g. across 3720\AA\ $< \lambda <$ 5680\AA, in
the $g$-band). However, we have taken advantage of the ongoing
Near-Earth Object WISE Reactivation mission \cite[NEOWISE-R;
][]{Mainzer2014, Meisner2017a, Meisner2017b}, as well as the Dark
Energy Camera Legacy Survey (DECaLS\footnote{{\tt
legacysurvey.org/decamls/}}) in order to discover new changing-look
quasars. While previous efforts have used the 1-year
baseline of the WISE mission to identify CLQs \cite[e.g., []{Assef2017}, our team is the first to extend .
Our team is the first to extend this selection to the
infrared using NEOWISE-R mission data, and we have identified a sample
of SDSS quasars that show dramatic decreases in their IR flux over the
course of a few years. These changes are on timescales too short to be
due to changes in obscuration, so a different explanation is needed.

In this article we present the $z=0.378$ quasar SDSS
J110057.70-005304.5 (hereafter J1100-0053) for which we have spectral
observations showing a transition in the blue-continuum slope
traditionally associated with the blackbody spectrum of an object with
broad hydrogen emission lines, into a `dim state' where the rest-frame
UV flux is suppressed, and then returning to a blue-continuum sloped
quasar.  We present a model that invokes changes at the ISCO to be the
triggering event for the change in the accretion disk, which along
with the changes in the broad emission lines, explains a major change
to the disk interior to 150$r_{g}$ (where $r_{\rm g}$ is the
gravitational radius; $r_{\rm g}=\frac{GM}{c^2}$) as well as the IR
light curves. Critically, our model makes predictions to the future
behaviour of J1100-0053.
 
\begin{figure}
  \centering
  %% trim=l b r t 
  % \includegraphics[width=8.00cm, height=7.50cm, trim=0.0cm 0.0cm 0.0cm 0.0cm, clip] {J110057_CRTS_lightcurve_v0pnt1.png}
  \includegraphics[width=16.00cm, height=10.00cm, trim=0.0cm 0.0cm 0.0cm 0.0cm, clip]
  {../plots/lc/J110057_lc_20171204v1.png}
  \caption[]{The light curve of J1100-0053. SDSS, DECaLS and PanSTARRS
    give the optical photometery. The WISE IR light curves are shown and
    their dramatic decrease led to the identification of J1100-0053. The
    three spectral epochs are shown by the vertical lines.}
  \label{fig:J110057_LC_CRTS}
\end{figure}


\begin{figure}
  \centering
  %% trim=l b r t 
  \includegraphics[width=17.00cm, height=12.00cm, trim=0.0cm 0.0cm 0.0cm 0.0cm, clip]
  {../plots/spectra/w1100m0052_sdss_wmodels.jpg}
  \caption[]{
Optical spectra of J1100-0053 obtained on MJD 51908 (blue; SDSS), 55302,
(red; BOSS) and 57809 (black; Palomar/DBSP).  {\it Left::} The full
optical spectra; {\it Right (top)::} Zoom in on the H$\beta$-\oiii
complex; {\it Right (bottom)::} Zoom in on the H$\alpha$-\nii complex.
}
  \label{fig:J110057_spectra}
\end{figure}
%%%%%%%%%%%%%%%%%%%%%%%%%%%%%%%%%%%%%%%%%%%%%%%%%%%%%%%%%%%%%
%%%%%%%%%%%%%%%%%%%%%%%%%%%%%%%%%%%%%%%%%%%%%%%%%%%%%%%%%%%%%
%%
%%   SECTION 2   SECTION 2   SECTION 2   SECTION 2   SECTION 2   SECTION 2  
%%   SECTION 2   SECTION 2   SECTION 2   SECTION 2   SECTION 2   SECTION 2  
%%   SECTION 2   SECTION 2   SECTION 2   SECTION 2   SECTION 2   SECTION 2  
%%
%%%%%%%%%%%%%%%%%%%%%%%%%%%%%%%%%%%%%%%%%%%%%%%%%%%%%%%%%%%%%
%%%%%%%%%%%%%%%%%%%%%%%%%%%%%%%%%%%%%%%%%%%%%%%%%%%%%%%%%%%%%
\section{Target Selection and Observations}  
We started by matching the SDSS-III BOSS Data Release 12 Quasar
catalog \cite[DR12Q; ][]{Paris2017} to the NEOWISE-R IR data (WISE W1
at 3.4$\mu$m, WISE W2 at 4.6$\mu$m). We found $\sim$200 objects
identified by a factor of 2 or more change in the observed WISE W1 and
W2 bands over the course of typically three or four years
\citep[see][and the Supplemental Material for the detailed NEOWISE-R
selection]{Meisner2017b}. Scanning these 200 objects, we also examined
the change in optical colour using the SDSS and DECaLS imaging surveys
in order to identify changes suggestive of changing-look quasars.
From this inspection, a priority list of $\approx70$ quasar targets
was derived and we obtained new optical spectroscopy from the Palomar
5m telescope.  J1100-0053 was one of these 70 objects, but critically,
had spectra from both SDSS and BOSS and was thus a priority target.

Figure~\ref{fig:J110057_LC_CRTS} presents the light curve of
J1100-0053.  Along with WISE IR data, optical data from the SDSS,
Catalina Real-time Transient Survey \citep[CRTS;][]{Drake2009,
Mahabal2011}, LINEAR \citep{Sesar2011} and PanSTARRS
\citep{Kaiser2010, Stubbs2010, Tonry2012, Magnier2013} are available.
Figure~\ref{fig:J110057_spectra} shows the three optical spectra of
J1100-0053 from the SDSS, BOSS and Palomar observations taken on MJD
51908 (UT 2000 December 30), 55302 (UT 2010 April 16) and 57809 (UT
2017 February 25), respectively.  The first-epoch SDSS spectrum shows
a typical blue quasar, but blue continuum then collapses in the second
epoch BOSS spectrum taken 10 years later. However, the blue continuum
has then returned in the third epoch spectrum 7 years later, albeit at
a diminished flux from the initial spectrum.  The Supplemental
Material gives further observation details including an achival ROSAT
detection.


%%%%%%%%%%%%%%%%%%%%%%%%%%%%%%%%%%%%%%%%%%%%%%%%%%%%%%%%%%%%%
%%%%%%%%%%%%%%%%%%%%%%%%%%%%%%%%%%%%%%%%%%%%%%%%%%%%%%%%%%%%%
%%
%%   SECTION 3   SECTION 3   SECTION 3   SECTION 3   SECTION 3   SECTION 3  
%%   SECTION 3   SECTION 3   SECTION 3   SECTION 3   SECTION 3   SECTION 3  
%%   SECTION 3   SECTION 3   SECTION 3   SECTION 3   SECTION 3   SECTION 3  
%%
%%%%%%%%%%%%%%%%%%%%%%%%%%%%%%%%%%%%%%%%%%%%%%%%%%%%%%%%%%%%%
%%%%%%%%%%%%%%%%%%%%%%%%%%%%%%%%%%%%%%%%%%%%%%%%%%%%%%%%%%%%%
\begin{figure*}
  %% trim=l b r t 
  \includegraphics[width=15.4cm, height=18.75cm, trim=0.0cm 0.0cm 0.0cm 0.0cm, clip]
  {../plots/models/cartoon_v2.pdf}
  \centering
  \caption[]{Our thermal model where an event at the ISCO causes a 
    cooling front to initiate and propagate out away from the central 
black hole. 
We predict that the flared disk will return to close its original state
in 2018.}
  \label{fig:J110057_diskmodel}
\end{figure*}
\section{Discussion}   
Our model of thermal emission from a multicolour disk implies 
changes in the region from the ISCO to $\sim$few tens-100 $r_{\rm g}$
are required to suppress flux into the observed
$g$-band. In particular, we suggest a physical collapse of the disk
scale height due to a cooling front propagating outward from the ISCO.

For J1100-0053 we apply our model as follows. We start with an inflated
disk, with non-zero torque at the ISCO, and $h/r\sim0.2$ inside of
$r\sim100 r_{\rm g}$.  This is the initial state circa 2000
($\approx$ MJD 51900).  Around 2007, there is a switch to a state of
zero torque at the ISCO. This causes a cooling front to propagate 
out from the ISCO over the timescale, $t_{\rm front}$. The drop in temperature 
leads to a drop in flux; regions behind
the cooling front emit 10\% of the flux compared to what they did prior to the
passage of the cooling front and the temperature decrease leads to the height 
dropping by a factor of 2.  $L_{\rm ion}$ starts decreasing due to the
drop in ionizing photons, which in turn causes the Balmer lines to 
decrease in flux.

If the inner accretion disk is usually inflated \cite[see e.g.,
][]{Sirko_Goodman2003, Thompson2005, Hopkins_Quataert2011}, such a
cooling front will naturally produce: 1) a collapse in the scale
height of the disk; 2) a decrease in flux moving from UV to longer
optical wavelengths; 3) a temporarily thicker scattering atmosphere,
further decreasing flux at short wavelengths.  This model implies
changes to the optical emission moving from shorter to longer
wavelengths (as the radius of the cooling front increases), on
months-to-years-long timescales. It also predicts a longer time to
recover the original flux (compared to the initial collapse) as a
front will move more slowly in a thinner disk (see Fig.~2). A decrease
in the UV flux would be expected to cause a decrease in IR flux, as
the heating of the IR-emitting dusty torus is reduced; however, there
should be a delay due to light travel time as well \cite[e.g.,
][]{Jun2015}.

Since the disk starts puffed-up, the cooling front time is not long,
and by 2010 (MJD 55300), the front has reached $r\sim50 r_{g}$. During
that time, the collapsing disk height increases the number density of
scatterers, which in turn causes Rayleigh scattering producing the
blue downturn in the 2010 spectrum.  The cooling front keeps going,
until it hits the part of the disk where it is normally thin, around
$r=100 r_g$, arriving around 2012. This sets up another (heating)
front, which will travel {\it back in} towards the SMBH, and
re-inflate the disk. This `returning' front travels more slowly
because the disk is thinner. This means the return to normal will be
asymmetric in time, as observed, and the $g$-band bottoms out first
because that is coming from $r\sim100r_{g}$ (see discussion in the
Supplemental Material).

Using \cite{Ford2018} and \cite{Sirko_Goodman2003},
Figure~\ref{fig:J110057_diskmodel} shows a model for a $M_{\rm
BH}=3\times 10^{8} M_{\odot}$, radiative efficiency of $\epsilon=0.1$,
accretion rate in units of Eddington accretion, $\dot{M}=0.032$, inner
disk radius of $6r_{\rm g}$ and outer disk radius of $10,000 r_{g}$. The resulting model spectra can be seen in
Figure~\ref{fig:J110057_diskmodel}.  We expect the front to return to
the ISCO in about 2018. That means the broad Balmer lines will come back a few
months later, but the WISE IR flux should not come back until about
2021.

\cite{Guo2016} observed a similar event to J1100-0053 with the source
SDSS J231742.60+000535.1. However, their object provided an ambiguous
case, as the IR brightness of their source did not decline. This is consistent with our model, as their cooling event is
relatively brief.  We discuss this object and the \cite{Guo2016} result
further in the Supplemental Material.



%%%%%%%%%%%%%%%%%%%%%%%%%%%%%%%%%%%%%%%%%%%%%%%%%%%%%%%%%%%%%
%%%%%%%%%%%%%%%%%%%%%%%%%%%%%%%%%%%%%%%%%%%%%%%%%%%%%%%%%%%%%
%%
%%   SECTION 4   SECTION 4   SECTION 4   SECTION 4   SECTION 4   SECTION 4  
%%   SECTION 4   SECTION 4   SECTION 4   SECTION 4   SECTION 4   SECTION 4  
%%   SECTION 4   SECTION 4   SECTION 4   SECTION 4   SECTION 4   SECTION 4  
%%
%%%%%%%%%%%%%%%%%%%%%%%%%%%%%%%%%%%%%%%%%%%%%%%%%%%%%%%%%%%%%
%%%%%%%%%%%%%%%%%%%%%%%%%%%%%%%%%%%%%%%%%%%%%%%%%%%%%%%%%%%%%
%\section{Method}

%\bibliographystyle{naturemag}
\bibliography{/cos_pc19a_npr/LaTeX/tester_mnras}
%\bibliography{sample}




\end{document}
