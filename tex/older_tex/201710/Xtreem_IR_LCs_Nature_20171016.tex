%%%%%%%%%%%%%%%%%%%%%%%%%%%%%%%%%%%%%%%%%%%%%%%%%%%%%%%%%%%%%
%%%%%%%%%%%%%%%%%%%%%%%%%%%%%%%%%%%%%%%%%%%%%%%%%%%%%%%%%%%%%
%%
%%   NPR's Nature template...  
%%   v1.0.0   Thu Dec  8 19:05:57 EST 2016
%%                                                                         
%%%%%%%%%%%%%%%%%%%%%%%%%%%%%%%%%%%%%%%%%%%%%%%%%%%%%%%%%%%%%
%%%%%%%%%%%%%%%%%%%%%%%%%%%%%%%%%%%%%%%%%%%%%%%%%%%%%%%%%%%%%
\documentclass{nature}
%\documentclass{natureprintstyle}

\bibliographystyle{naturemag}

\input{format}

\usepackage{mathptmx}
\usepackage[T1]{fontenc}
%\usepackage[square,sort,comma,numbers]{natbib}
\usepackage{amsmath}
\usepackage{amssymb}
\usepackage{graphicx}
\usepackage{threeparttable}


%\title{A Quasar caught in the act of turning off}
\title{A new interpretation of optical and infrared variability in Quasars}

\author{Nicholas~P.~Ross$^{1}$ et al.}  
% Saavik~K.~Ford,$^{7}$, Barry~L.~McKernan$^{7}$,
% Daniel~Stern$^{3}$, Aaron~Meisner$^{2}$, 
% Matthew~Graham$^{4}$, 
% Arjun~Dey$^{5}$ \& David~J.~Schlegel$^{2}$ 
% Andrew~Drake$^{4}$, 
% Roberto~J.~Assef$^{6},
% Mislav~Balokovic$^{6}, 
% Murray~Brightman$^{4}, 
% for the DECaLS Consortium. 


\begin{document}

\maketitle

\begin{affiliations}
  \item Institute for Astronomy, University of Edinburgh, Royal Observatory, Blackford Hill, Edinburgh EH9 3HJ, United Kingdom 
 % \item Department of Astrophysics is located in the Rose Center for Earth and Space, American Museum of Natural History, Central Park West at 79th Street, New York, NY 10024, U.S.A
  %\item Jet Propulsion Laboratory, California Institute of Technology, 4800 Oak Grove Drive, Mail Stop 169-221, Pasadena, CA 91109, USA 
  %\item California Institute of Technology, 1200 East California Boulevard, Pasadena, CA 91125, USA
  %\item Steward Observatory, 933 North Cherry Avenue, Tucson, AZ 85721, U.S.A.
%  \item Universidad Diego Portales, Av Republica 180, Santiago, Regi ́on Metropolitana, Chile}
%  \item Lawrence Berkeley National Laboratory, 1 Cyclotron Road, Berkeley, CA 92420, U.S.A. 
\end{affiliations}


\begin{abstract}
Changing-look quasars are a class of recently identified object in
which the strong UV continuum and broad optical hydrogen emission
lines associated with unobscured quasars either appear or disappear on
timescales of months to years \cite{LaMassa15, Runnoe16, MacLeod16, Ruan16}. The
physical processes responsible for this behaviour are still debated,
but changes in the black hole accretion rate or accretion disk
structure appear more likely than changes in
obscuration \cite{Hutsemekers17, Sheng17}. Here we report on three
epochs of spectroscopy of SDSS J110057.70-005304.5, a quasar whose UV
continuum and broad hydrogen emission lines have dramatically faded
over the past 20 years. An archival spectrum of this quasar from 2010
shows an intermediate phase of the transition during which the flux
below rest-frame 340nm has collapsed. This is unique compared to
previously published examples of changing-look quasars, and is best
explained by dramatic changes in the innermost regions of the
accretion disk. The optical continuum has been rising again since
mid-2016, leading to a prediction of a rise in hydrogen emission line
flux in the next few months. If our model is confirmed, the physics of
`changing look' quasars are governed by processes at the innermost
stable circular orbit (ISCO) around the black hole, and the structure
of the innermost disk. Thus, the easily identifiably and monitored
Changing Look Quasars would then provide a new probe of the strong
gravity regime.
\end{abstract}



%%%%%%%%%%%%%%%%%%%%%%%%%%%%%%%%%%%%%%%%%%%%%%%%%%%%%%%%%%%%%
%%%%%%%%%%%%%%%%%%%%%%%%%%%%%%%%%%%%%%%%%%%%%%%%%%%%%%%%%%%%%
%%
%%   SECTION 1  SECTION 1  SECTION 1  SECTION 1  SECTION 1  SECTION 1  
%%   SECTION 1  SECTION 1  SECTION 1  SECTION 1  SECTION 1  SECTION 1  
%%   SECTION 1  SECTION 1  SECTION 1  SECTION 1  SECTION 1  SECTION 1  
%%
%%%%%%%%%%%%%%%%%%%%%%%%%%%%%%%%%%%%%%%%%%%%%%%%%%%%%%%%%%%%%
%%%%%%%%%%%%%%%%%%%%%%%%%%%%%%%%%%%%%%%%%%%%%%%%%%%%%%%%%%%%%
%\section{Introduction}

The ``Changing-Look'' quasar phenomenon, where the dramatic
disappearance, or appearance, of prominent broad optical emission
lines is seen on year timescales, is now widely observed,
\cite{LaMassa15, MacLeod16, Runnoe16, Ruan16, Gezari17, Rumbaugh17}.
yet poorly understood. Changes in obscuration are generally disfavoured, 
\cite{Hutsemekers17, Sheng17}, and it is clear that the CLQs are a
key laboratory into understanding accretion physics and the nature of
the AGN broad line region (BLR).
%%
Zooming in on the AGN accretion disk, the famous $\alpha-$disk model \cite{SS73} for a optically thick,
geometrically thin disk ($h / R \ll 1$, where $h$ is the vertical
scale height of the disk) is known to have serious short-comings
e.g. \cite{Antonucci99, Koratkar_Blaes99, Lawrence12} AGN seem to be
cooler than they ought to be \cite[e.g., ][]{Lawrence12} with the SEDs
of AGN showing a universal near-UV shape, reaching a maximum in $\nu
S_{\nu}$ around 1100\AA\ , Such a peak suggests a characteristic
temperature of T$\sim$30 000K, wheres for a thermal model, the
characteristic temperature should be roughly T$\sim$100 000K.
Moreover, constraints from microlensing observartions for the size of
the optical emission region \cite[e.g., ]{Pooley07, Morgan10,
Morgan12, Mosquera11} suggestion this region is larger than the one
predicted by the standard Shakura-Sunyaev disk.

CLQs have traditionally been discovered by looking for large, $|
\Delta m | >1$ magnitude changes in the optical light curves (e.g. in
the $g$-band). However, we have taken advantage of the ongoing
Near-Earth Object WISE Reactivation mission
(NEOWISE-R)\cite{Mainzer14, Meisner17, Meisner17b}, as well as the
Dark Energy Camera Legacy Survey (DECaLS\footnote{{\tt
legacysurvey.org/decamls/}}) in order to discover new CLQs. Our team
is the first to extend this selection to the infrared using NEOWISE-R
mission data. Indeed, we have found a sample of SDSS quasars that show
{\it dramatic decreases in their IR flux over the course of a few
years.}  These changes are on timescales too short to be considered
due to changes in obscuration, so a new explanation is needed.

In this article we present the $z=0.378$ quasar SDSS
J110057.70-005304.5 that we have observed transitioning from a blue
continuum sloped object to become a regular galaxy. However, along
with the changes in the BELs, we see a major change to the disk
interior to 150$R_{g}$.
 
\begin{figure*}
  \centering
  %% trim=l b r t 
  % \includegraphics[width=8.00cm, height=7.50cm, trim=0.0cm 0.0cm 0.0cm 0.0cm, clip] {J110057_CRTS_lightcurve_v0pnt1.png}
  \includegraphics[width=16.00cm, height=7.00cm, trim=0.0cm 0.0cm 0.0cm 0.0cm, clip]
  {../plots/lc/J110057_lc_20171016.pdf}
  \caption[]{The light curve of J110057. SDSS, DECaLS and PanSTARRS
    give the optical photometery. The WISE IR light curves are shown and
    their dramatic decrease led to the identification of J110057. The
    three spectral epochs are shown by the vertical lines.}
  \label{fig:J110057_LC_CRTS}
\end{figure*}


\begin{figure*}
  %% trim=l b r t 
  \includegraphics[width=16.00cm, height=7.00cm, trim=0.0cm 0.0cm 0.0cm 0.0cm, clip]
  {../plots/spectra/J110057_spectra_wBalmers.pdf}
  \centering
  \caption[]{
Optical spectra of J110057 obtained on MJD 51908 (blue; SDSS), 55302,
(red; BOSS) and 57809 (black; Palomar/DBSP).  {\it Left::} The full
optical spectra; {\it Right (top)::} Zoom in on the H$\beta$-\oiii
complex; {\it Right (bottom)::} Zoom in on the H$\alpha$-\nii complex.
}
  \label{fig:J110057_spectra}
\end{figure*}
%%%%%%%%%%%%%%%%%%%%%%%%%%%%%%%%%%%%%%%%%%%%%%%%%%%%%%%%%%%%%
%%%%%%%%%%%%%%%%%%%%%%%%%%%%%%%%%%%%%%%%%%%%%%%%%%%%%%%%%%%%%
%%
%%   SECTION 2   SECTION 2   SECTION 2   SECTION 2   SECTION 2   SECTION 2  
%%   SECTION 2   SECTION 2   SECTION 2   SECTION 2   SECTION 2   SECTION 2  
%%   SECTION 2   SECTION 2   SECTION 2   SECTION 2   SECTION 2   SECTION 2  
%%
%%%%%%%%%%%%%%%%%%%%%%%%%%%%%%%%%%%%%%%%%%%%%%%%%%%%%%%%%%%%%
%%%%%%%%%%%%%%%%%%%%%%%%%%%%%%%%%%%%%%%%%%%%%%%%%%%%%%%%%%%%%
\section{Results}  
%% \subsection(Results subheadings, if needed}
Matching the SDSS/BOSS Data Release 12 Quasar catalog ( \href{DR12Q;
}{Paris17)} to the NEOWISE-R IR data (W1 is 3.4$\mu$m, W2 is
4.6$\mu$m) our team found $\approx$200 objects with fading light IR
light curves. These objects were identified by a factor of 2 or more
drop in the observed WISE W1 and W2 bands.

Figure~\ref{fig:J110057_LC_CRTS} gives the optical light curve of
J110057.  Figure~\ref{fig:J110057_spectra} shows the three optical
spectra of J110057. 

Checking the data archives we found there was 
no source within 30 arcsec in the VLA FIRST, i.e., at 21 cm radio frequencies.
%catalog detection limit (including CLEAN bias) at source position is 1.11 mJy/beam
None of the {\it Hubble Space Telescope}, the {\it Spitzer Space
Telescope} or the {\it Kepler} Mission has observed J110057 patch of
sky. It is also not in the HSC DR1 footprint.  There is a detection in
ROSAT, using the 2nd all-sky survey (2RXS; Boller et al. 2016, A\&A,
588, 103) as 2RXS J110058.1-005259 with 27.00 counts (count error
6.14), count rate= 0.06$\pm$0.01 and an exposure time of = 431.95
seconds.  The NED gives J110057 as $1.27\pm0.28 \times^{-12}$
erg/cm$^{2}$/s in the 0.1-2.4 keV range (unabsorbed
flux). J110057Neither {\it Chandra} or {\it XMM-Newton}.

Figure~\ref{fig:J110057_spectra} shows the optical spectra for J110057
from the SDSS, the BOSS and Palomar observations taken on MJD 51908,
55302, and 57809. The MJD 51908 SDSS spectrum is of a typical blue quasar, 
the blue continuum then collapses 



\section{Discussion}   %% (without subheadings)
Our preliminary modelsof emission from a multicolor disk (Fig.~3) 
imply changes from the innermost stable circular orbit (ISCO) to
$\sim$few tens-100 $R_{\rm g}$ are required to suppress flux into the
observed $g$-band. In particular, we suggest a physical collapse of
the disk scale height due to a cooling front propagating outward from
the ISCO. 
%%
If the inner accretion disk is usually inflated (see
e.g. Sirko \& Goodman 2003, Thompson et al. 2005, Hopkins \& Quataert
2011), such a cooling front will naturally produce: 1) a collapse in
the scale height of the disk; 2) a decrease in flux moving from UV to
longer optical wavelengths; 3) a temporarily thicker scattering
atmosphere, further decreasing flux at short wavelengths.  This
model implies changes to the optical emission moving from shorter to
longer wavelengths (as the radius of the cooling front increases), on
months-to-years-long timescales. It also predicts a longer time to
recover the original flux (compared to the initial collapse, as a
front will move more slowly in a thinner disk (see Fig.~2). A decrease
in the UV flux would also be expected to cause a decrease in IR flux,
as the heating of the IR-emitting dusty torus is reduced; however,
there should also be a delay due to light travel time.


\begin{figure*}
  %% trim=l b r t 
  \includegraphics[width=15.4cm, height=8.75cm, trim=0.0cm 0.0cm 0.0cm 0.0cm, clip]
  {../plots/models/mcd_gap_v3_20171016v1.pdf}
  \centering
  \caption[]{
{\bf PLACE HOLDER OF OBSERVED AND MODEL FOR J110057. }
}
  \label{fig:J110057_diskmodel}
\end{figure*}

For J110057 we apply our model as follows. We start with an inflated
disk, with non-zero torque at the ISCO, and $h/R\sim0.2$ inside of
$R\sim100 R_{\rm g}$.  This is the initial state in circa 2000
($\approx$ MJD 51900).  Around 2007, there is a switch to a zero torque
at ISCO state.  A cooling front is set-up, which propagates out from
the ISCO at the timescale, $t_{\rm front}$. Regions behind the front
emit flux at $\sim$0.1$L$ of what they did prior to the passage of the
front ({\bf NPR} comment;: due to drop in T?  T $\downarrow \times
1.78 \Rightarrow L \downarrow \times10$) , and the temperature
decrease leads the height to drop by a factor of 2.  ({\bf NPR}
comment: just due to less kinetic energy??).  $L_{\rm ion}$ starts
decreasing due to the drop in ionizing photons, which in turn causes
the Balmer lines to also decrease in flux.

Since the disk starts puffed-up, the cooling front time is not long,
and by 2010 (MJD 55300), the front has reached $R\sim50 R_{g}$. During
that time, the collapsing disk height increases the number density of
scatterers, which in turn causes Rayleigh scattering producing the
blue downturn in the 2010 spectrum.  The cooling front keeps going,
until it hits the part of the disk where it is normally thin, around
$R=100 R_g$, arriving around 2012. This sets up another (heating)
front, which will travel {\it back in} towards the SMBH, and
re-inflate the disk. This 'returning' front travels more slowly
because the disk is thinner. It also means the return to normal will
be asymmetric in time, as observed, and the $g$-band bottoms out first
because that is coming from $R\sim100R_{g}$.

Using Ford et al and Sirko \& Goodman 2003, Figure~\ref{fig:J110057_diskmodel}
shows a model for a $M_{\rm BH}=3\times 10^{8} M_{\odot}$, radiative
efficiency of $\epsilon=0.1$, accretion rate in units of Eddington
accretion, $\dot{M}=0.032$, inner and outer disk radii in units of
$r_g$ of SMBH of radius$_{\rm in}$=6.0, radius$_{\rm out}$=1.0$\times
10^{4}$. The resulting model spectra can be seen in
Figure~\ref{fig:J110057_diskmodel}.

We expect the front to return to the ISCO in about 2018. That
means the H lines will come back a few months later, but the WISE IR
flux shouldn't come back until about 2021.

\cite{Guo16} observed a similar event to J110057, with SDSS
J231742.60 +000535.1. However, their object provided an ambiguous
case, as the IR brightness of their source did not decline. However
this is consistent with our model, as their cooling event is
relatively brief.




%%%%%%%%%%%%%%%%%%%%%%%%%%%%%%%%%%%%%%%%%%%%%%%%%%%%%%%%%%%%%
%%%%%%%%%%%%%%%%%%%%%%%%%%%%%%%%%%%%%%%%%%%%%%%%%%%%%%%%%%%%%
%%
%%   SECTION 2   SECTION 2   SECTION 2   SECTION 2   SECTION 2   SECTION 2  
%%   SECTION 2   SECTION 2   SECTION 2   SECTION 2   SECTION 2   SECTION 2  
%%   SECTION 2   SECTION 2   SECTION 2   SECTION 2   SECTION 2   SECTION 2  
%%
%%%%%%%%%%%%%%%%%%%%%%%%%%%%%%%%%%%%%%%%%%%%%%%%%%%%%%%%%%%%%
%%%%%%%%%%%%%%%%%%%%%%%%%%%%%%%%%%%%%%%%%%%%%%%%%%%%%%%%%%%%%
%\section{Method}



%\bibliographystyle{naturemag}
%bibliography{/cos_pc19a_npr/LaTeX/tester_mnras}
%\bibliography{sample}



\begin{thebibliography}{10}
\expandafter\ifx\csname url\endcsname\relax
  \def\url#1{\texttt{#1}}\fi
\expandafter\ifx\csname urlprefix\endcsname\relax\def\urlprefix{URL }\fi
\providecommand{\bibinfo}[2]{#2}
\providecommand{\eprint}[2][]{\url{#2}}

\bibitem{LaMassa15}
\bibinfo{author}{{LaMassa}, S.~M.} \emph{et~al.}
\newblock \bibinfo{title}{{The Discovery of the First ``Changing Look'' Quasar:
  New Insights Into the Physics and Phenomenology of Active Galactic Nucleus}}.
\newblock \emph{\bibinfo{journal}{\apj}} \textbf{\bibinfo{volume}{800}},
  \bibinfo{pages}{144} (\bibinfo{year}{2015}).
\newblock \eprint{1412.2136}.

\bibitem{Runnoe16}
\bibinfo{author}{{Runnoe}, J.~C.} \emph{et~al.}
\newblock \bibinfo{title}{{Now you see it, now you don't: the disappearing
  central engine of the quasar J1011+5442}}.
\newblock \emph{\bibinfo{journal}{\mnras}} \textbf{\bibinfo{volume}{455}},
  \bibinfo{pages}{1691--1701} (\bibinfo{year}{2016}).
\newblock \eprint{1509.03640}.

\bibitem{MacLeod16}
\bibinfo{author}{{MacLeod}, C.~L.} \emph{et~al.}
\newblock \bibinfo{title}{{A systematic search for changing-look quasars in
  SDSS}}.
\newblock \emph{\bibinfo{journal}{\mnras}} \textbf{\bibinfo{volume}{457}},
  \bibinfo{pages}{389--404} (\bibinfo{year}{2016}).
\newblock \eprint{1509.08393}.

\bibitem{Ruan16}
\bibinfo{author}{{Ruan}, J.~J.} \emph{et~al.}
\newblock \bibinfo{title}{{Toward an Understanding of Changing-look Quasars: An
  Archival Spectroscopic Search in SDSS}}.
\newblock \emph{\bibinfo{journal}{\apj}} \textbf{\bibinfo{volume}{826}},
  \bibinfo{pages}{188} (\bibinfo{year}{2016}).
\newblock \eprint{1509.03634}.

\bibitem{Hutsemekers17}
\bibinfo{author}{{Hutsem{\'e}kers}, D.}, \bibinfo{author}{{Ag{\'{\i}}s
  Gonz{\'a}lez}, B.}, \bibinfo{author}{{Sluse}, D.}, \bibinfo{author}{{Ramos
  Almeida}, C.} \& \bibinfo{author}{{Acosta Pulido}, J.-A.}
\newblock \bibinfo{title}{{Polarization of the changing-look quasar
  J1011+5442}}.
\newblock \emph{\bibinfo{journal}{\aap}} \textbf{\bibinfo{volume}{604}},
  \bibinfo{pages}{L3} (\bibinfo{year}{2017}).
\newblock \eprint{1707.05540}.

\bibitem{Sheng17}
\bibinfo{author}{{Sheng}, Z.} \emph{et~al.}
\newblock \bibinfo{title}{{Mid-infrared Variability of Changing-look AGNs}}.
\newblock \emph{\bibinfo{journal}{\apjl}} \textbf{\bibinfo{volume}{846}},
  \bibinfo{pages}{L7} (\bibinfo{year}{2017}).
\newblock \eprint{1707.02686}.

\bibitem{Gezari17}
\bibinfo{author}{{Gezari}, S.} \emph{et~al.}
\newblock \bibinfo{title}{{iPTF Discovery of the Rapid ``Turn-on'' of a
  Luminous Quasar}}.
\newblock \emph{\bibinfo{journal}{\apj}} \textbf{\bibinfo{volume}{835}},
  \bibinfo{pages}{144} (\bibinfo{year}{2017}).
\newblock \eprint{1612.04830}.

\bibitem{Rumbaugh17}
\bibinfo{author}{{Rumbaugh}, N.} \emph{et~al.}
\newblock \bibinfo{title}{{Extreme variability quasars from the Sloan Digital
  Sky Survey and the Dark Energy Survey}}.
\newblock \emph{\bibinfo{journal}{ArXiv e-prints}}  (\bibinfo{year}{2017}).
\newblock \eprint{1706.07875}.

\bibitem{SS73}
\bibinfo{author}{{Shakura}, N.~I.} \& \bibinfo{author}{{Sunyaev}, R.~A.}
\newblock \bibinfo{title}{{Black holes in binary systems. Observational
  appearance.}}
\newblock \emph{\bibinfo{journal}{\aap}} \textbf{\bibinfo{volume}{24}},
  \bibinfo{pages}{337} (\bibinfo{year}{1973}).

\bibitem{Antonucci99}
\bibinfo{author}{{Antonucci}, R.}
\newblock \bibinfo{title}{{Constraints on Disks Models of The Big Blue Bump
  from UV/Optical/IR Observations}}.
\newblock In \bibinfo{editor}{{Poutanen}, J.} \& \bibinfo{editor}{{Svensson},
  R.} (eds.) \emph{\bibinfo{booktitle}{High Energy Processes in Accreting Black
  Holes}}, vol. \bibinfo{volume}{161} of \emph{\bibinfo{series}{Astronomical
  Society of the Pacific Conference Series}}, \bibinfo{pages}{193}
  (\bibinfo{year}{1999}).
\newblock \eprint{astro-ph/9810067}.

\bibitem{Koratkar_Blaes99}
\bibinfo{author}{{Koratkar}, A.} \& \bibinfo{author}{{Blaes}, O.}
\newblock \bibinfo{title}{{The Ultraviolet and Optical Continuum Emission in
  Active Galactic Nuclei: The Status of Accretion Disks}}.
\newblock \emph{\bibinfo{journal}{\pasp}} \textbf{\bibinfo{volume}{111}},
  \bibinfo{pages}{1--30} (\bibinfo{year}{1999}).

\bibitem{Lawrence12}
\bibinfo{author}{{Lawrence}, A.}
\newblock \bibinfo{title}{{The UV peak in active galactic nuclei: a false
  continuum from blurred reflection?}}
\newblock \emph{\bibinfo{journal}{\mnras}} \textbf{\bibinfo{volume}{423}},
  \bibinfo{pages}{451--463} (\bibinfo{year}{2012}).
\newblock \eprint{1110.0854}.

\bibitem{Pooley07}
\bibinfo{author}{{Pooley}, D.}, \bibinfo{author}{{Blackburne}, J.~A.},
  \bibinfo{author}{{Rappaport}, S.} \& \bibinfo{author}{{Schechter}, P.~L.}
\newblock \bibinfo{title}{{X-Ray and Optical Flux Ratio Anomalies in Quadruply
  Lensed Quasars. I. Zooming in on Quasar Emission Regions}}.
\newblock \emph{\bibinfo{journal}{\apj}} \textbf{\bibinfo{volume}{661}},
  \bibinfo{pages}{19--29} (\bibinfo{year}{2007}).
\newblock \eprint{astro-ph/0607655}.

\bibitem{Morgan10}
\bibinfo{author}{{Morgan}, C.~W.}, \bibinfo{author}{{Kochanek}, C.~S.},
  \bibinfo{author}{{Morgan}, N.~D.} \& \bibinfo{author}{{Falco}, E.~E.}
\newblock \bibinfo{title}{{The Quasar Accretion Disk Size-Black Hole Mass
  Relation}}.
\newblock \emph{\bibinfo{journal}{\apj}} \textbf{\bibinfo{volume}{712}},
  \bibinfo{pages}{1129--1136} (\bibinfo{year}{2010}).
\newblock \eprint{1002.4160}.

\bibitem{Morgan12}
\bibinfo{author}{{Morgan}, C.~W.} \emph{et~al.}
\newblock \bibinfo{title}{{Further Evidence that Quasar X-Ray Emitting Regions
  are Compact: X-Ray and Optical Microlensing in the Lensed Quasar Q
  J0158-4325}}.
\newblock \emph{\bibinfo{journal}{\apj}} \textbf{\bibinfo{volume}{756}},
  \bibinfo{pages}{52} (\bibinfo{year}{2012}).
\newblock \eprint{1205.4727}.

\bibitem{Mosquera11}
\bibinfo{author}{{Mosquera}, A.~M.} \& \bibinfo{author}{{Kochanek}, C.~S.}
\newblock \bibinfo{title}{{The Microlensing Properties of a Sample of 87 Lensed
  Quasars}}.
\newblock \emph{\bibinfo{journal}{\apj}} \textbf{\bibinfo{volume}{738}},
  \bibinfo{pages}{96} (\bibinfo{year}{2011}).
\newblock \eprint{1104.2356}.

\bibitem{Mainzer14}
\bibinfo{author}{{Mainzer}, A.} \emph{et~al.}
\newblock \bibinfo{title}{{Initial Performance of the NEOWISE Reactivation
  Mission}}.
\newblock \emph{\bibinfo{journal}{\apj}} \textbf{\bibinfo{volume}{792}},
  \bibinfo{pages}{30} (\bibinfo{year}{2014}).
\newblock \eprint{1406.6025}.

\bibitem{Meisner17}
\bibinfo{author}{{Meisner}, A.~M.}, \bibinfo{author}{{Lang}, D.} \&
  \bibinfo{author}{{Schlegel}, D.~J.}
\newblock \bibinfo{title}{{Deep Full-sky Coadds from Three Years of WISE and
  NEOWISE Observations}}.
\newblock \emph{\bibinfo{journal}{\aj}} \textbf{\bibinfo{volume}{154}},
  \bibinfo{pages}{161} (\bibinfo{year}{2017}).
\newblock \eprint{1705.06746}.

\bibitem{Meisner17b}
\bibinfo{author}{{Meisner}, A.~M.} \emph{et~al.}
\newblock \bibinfo{title}{{Searching for Planet Nine with Coadded WISE and
  NEOWISE-Reactivation Images}}.
\newblock \emph{\bibinfo{journal}{\aj}} \textbf{\bibinfo{volume}{153}},
  \bibinfo{pages}{65} (\bibinfo{year}{2017}).
\newblock \eprint{1611.00015}.

\bibitem{Guo16}
\bibinfo{author}{{}, H.} \emph{et~al.}
\newblock \bibinfo{title}{{The Optical Variability of SDSS Quasars from
  Multi-epoch Spectroscopy. III. A Sudden UV Cutoff in Quasar SDSS
  J2317+0005}}.
\newblock \emph{\bibinfo{journal}{\apj}} \textbf{\bibinfo{volume}{826}},
  \bibinfo{pages}{186} (\bibinfo{year}{2016}).
\newblock \eprint{1605.07301}.

\end{thebibliography}



\end{document}
