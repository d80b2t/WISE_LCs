%%%%%%%%%%%%%%%%%%%%%%%%%%%%%%%%%%%%%%%%%%%%%%%%%%%%%%%%%%%%%
%%
%%
%%     NPR's  Nature template...  
%%     v1.0.0   Thu Dec  8 19:05:57 EST 2016
%%                                                                         
%%
%%%%%%%%%%%%%%%%%%%%%%%%%%%%%%%%%%%%%%%%%%%%%%%%%%%%%%%%%%%%%
\documentclass{nature}
%\documentclass{natureprintstyle}

\usepackage{graphicx, fancyhdr, subfigure}
\usepackage{epsfig, psfig, epsf}
\usepackage{amsmath, amssymb, cancel, mathptmx}
\usepackage[T1]{fontenc}
\usepackage{dcolumn}  %%  Align table columns on decimal point
\usepackage{bm}           %%  bold math
\usepackage{hyperref,ifthen}
\usepackage{verbatim, threeparttable}
\usepackage[square,sort,comma,numbers]{natbib}

\newcommand{\oiii}{[O\,{\sc iii}]\ }
\newcommand{\nii}{N\,{\sc ii}\ }


\bibliographystyle{naturemag}

\title{A new physical interpretation of optical and infrared variability in quasars}

\author{Nicholas~P.~Ross$^{1}$ et al.  
%K. E. Saavik Ford$^{2,3,4}$
%Matthew Graham$^{5}$
%Barry McKernan$^{2,3,4}$
%Daniel Stern$^{5}$, 
%Aaron Meisner, 
%Andrew Drake, 
%Arjun Dey, 
%David Schlegel, 
%Roberto Assef, 
%Hyunsung Jun, 
%Peter Eisenhardt, 
%George Djorgovsk
}

\begin{document}

\maketitle

\begin{affiliations}
  \item Institute for Astronomy, University of Edinburgh, Royal Observatory, Blackford Hill, Edinburgh EH9 3HJ, United Kingdom 
%  \item Department of Science, BMCC, City University of New York, New York, NY 10007, USA
 % \item Department of Astrophysics, American Museum of Natural History, New York, NY 10024, USA
 % \item  Graduate Center, City University of New York, 365 5th Avenue, New York, NY 10016, USA
%XSXS\item Cahill Center for Astronomy and Astrophysics, California Institute of Technology, Mail Code 249/17, 1200 E California Blvd, Pasadena CA 91125, USA
 % \item Department of Astrophysics is located in the Rose Center for Earth and Space, American Museum of Natural History, Central Park West at 79th Street, New York, NY 10024, U.S.A
  %\item Jet Propulsion Laboratory, California Institute of Technology, 4800 Oak Grove Drive, Mail Stop 169-221, Pasadena, CA 91109, USA 
  %\item California Institute of Technology, 1200 East California Boulevard, Pasadena, CA 91125, USA
  %\item Steward Observatory, 933 North Cherry Avenue, Tucson, AZ 85721, U.S.A.
%  \item Universidad Diego Portales, Av Republica 180, Santiago, Regi ́on Metropolitana, Chile}
%  \item Lawrence Berkeley National Laboratory, 1 Cyclotron Road, Berkeley, CA 92420, U.S.A. 
\end{affiliations}


\begin{abstract}
Changing-look quasars are a recently identified class of active galaxies in which the strong UV continuum and/or broad optical hydrogen emission lines associated with unobscured quasars either appear or disappear on timescales of months to years \cite{LaMassa2015, Runnoe2016, MacLeod2016, Ruan2016, Yang2017}.  The physical processes responsible for this behaviour are still debated, but changes in the black hole accretion rate or accretion disk structure appear more likely than changes in obscuration \cite{Hutsemekers2017, Sheng2017}.  Here we report on three epochs of spectroscopy of SDSS J110057.70-005304.5, a quasar at a redshift of $z=0.378$ whose UV continuum and broad hydrogen emission lines have dramatically faded over the past $\approx$20 years. The change in this quasar was initially identified in the infrared, and an archival spectrum from 2010 shows an intermediate phase of the transition during which the flux below rest-frame 3400\AA\ has collapsed. This combination is unique compared to previously published examples of changing-look quasars, and is best explained by dramatic changes in the innermost regions of the accretion disk. The optical continuum has been rising again since mid-2016, leading to a prediction of a rise in hydrogen emission line flux in the next year. If our model is confirmed, the physics of changing-look quasars are governed by processes at the innermost stable circular orbit (ISCO) around the black hole, and the structure of the innermost disk. The easily identifiable and monitored changing-look quasars would then provide a new probe and laboratory of the nuclear central engine. 
\end{abstract}



%%%%%%%%%%%%%%%%%%%%%%%%%%%%%%%%%%%%%%%%%%%%%%%%%%%%%%%%%%%%%%%%%%%%%%%%%%%%%%%%%
%%%%%%%%%%%%%%%%%%%%%%%%%%%%%%%%%%%%%%%%%%%%%%%%%%%%%%%%%%%%%%%%%%%%%%%%%%%%%%%%%
%%
%%
%%   SECTION 1  SECTION 1  SECTION 1  SECTION 1  SECTION 1  SECTION 1  
%%   SECTION 1  SECTION 1  SECTION 1  SECTION 1  SECTION 1  SECTION 1  
%%   SECTION 1  SECTION 1  SECTION 1  SECTION 1  SECTION 1  SECTION 1  
%%
%%
%%%%%%%%%%%%%%%%%%%%%%%%%%%%%%%%%%%%%%%%%%%%%%%%%%%%%%%%%%%%%%%%%%%%%%%%%%%%%%%%%%
%%%%%%%%%%%%%%%%%%%%%%%%%%%%%%%%%%%%%%%%%%%%%%%%%%%%%%%%%%%%%%%%%%%%%%%%%%%%%%%%%%
%\section{Introduction}

%% CLQ phenomenon introduced.  a little more history, a little more on what was found and learned from the optical variability selection. maybe connect to CL-Seyferts (or not).  and then end with the extension to mid-IR variability selection in general, and this source in specific.  emphasize that previous CLQ work has treated the physical changes of the accretion disk at a very cursory level, merely invoking such changes in the accretion disk without a physical model. here we're using this one interesting CLQ to develop a new physical interpretation of the variability.  maybe not that different than what  you already have, and maybe no changes necessary.  i guess one possible change, however, is that in a Nature paper, being so short, you don't need to get into the UV collapse in the Intro; I'd save that for the next section -- only 3 paragraphs later (and, appropriately so, you've already mentioned the UV collapse in the abstract). - alternatively, that CLQ intro I suggest above is basically our historical entry into this target.  the other approach is to forget about the history and focus on what we're learning.  in that vein, maybe the intro would emphasize accretion disk models, their strengths and weaknesses, and then say how time variability offers a unique, important, new way to study the still very uncertain physics of what's going on near SMBH's.  in generally, the latter approach where you forget about your history of studying a topic and jump to what you've learned is the preferred approach. but that would be a major shift compared to the current text.

The changing-look quasar phenomenon, where the dramatic disappearance, or appearance, of strong UV continuum and/or prominent broad emission lines is seen on month-to-year timescales, is now widely observed \cite{LaMassa2015, MacLeod2016, Runnoe2016, Ruan2016, Gezari2017, Rumbaugh2017, Yang2017}, yet poorly understood. Changes in obscuration are generally disfavoured due to the timescales currently observed \cite{Hutsemekers2017, Sheng2017}, and it is clear that changing-look quasars are a key laboratory for understanding accretion physics and active galactic nuclei (AGN).

Changing-look quasars have traditionally been discovered by looking for large, $| \Delta m | >1$ magnitude changes in the optical light curves of quasars or galaxies. In contrast, we have taken advantage of the ongoing mid-infrared Near-Earth Object WISE Reactivation mission \cite[NEOWISE-R; ][]{Mainzer2014, Meisner2017a, Meisner2017b}, supplemented with the optical Dark Energy Camera Legacy Survey (DECaLS\footnote{{\tt legacysurvey.org/decamls/}}) in order to discover new changing-look quasars. While previous efforts have used the 1-year baseline of the WISE mission to identify CLQs \cite[e.g.,][]{Assef2017}, our investigation is the first to extend this selection to the infrared using NEOWISE-R mission data and we have identified a sample of Sloan Digital Sky Survey (SDSS) quasars that show significant changes in their IR flux over the course of a few years.

The Shakura-Sunyaev $\alpha-$disk model \cite{SS73} has long been used to (over-)simply describe the basic properties of the optically thick, geometrically thin accretion disks expected to orbit the supermassive black holes at the nuclei of quasars.  This accretion disk is thought to be the origin of thermal continuum emission that is observed in the rest-frame ultraviolet and optical.  The key assumptions are that the state of the plasma within the disk is optically thick and thermal, and that the accretion disk can be treated as a blackbody. Then, given the size scales associated with supermassive black holes and the associated temperatures, this implies that a substantial fraction of the bolometric luminosity should be in the form of UV photons. The thermal emission seen in the infrared is believed to originate in the dusty, molecular torus, representing emission from the disk or broad line region (BLR) that has been reprocessed in this optically thick region into thermal radiation \citep[see ][for reviews]{Antonucci1993, Perlman2008, Lasota2016}.

While the mass of the black hole is large compared to the accretion disk, the angular momentum of the black hole hole is small compared to the accretion disk.  Hence angular momentum has to be lost from the accretion disk.  A kinematic viscosity seems the likely mechanism that angular momentum is transported outward.  This viscosity is likely due to magnetorotational instability \citep[MRI; ][]{Balbus_Hawley1991} with additional contributions to turbulence from the effects of embedded objects in the disk e.g., \cite{McKernan2014}.  However, as e.g., \cite{Koratkar_Blaes1999, Sirko_Goodman2003} point out, the observed spectral energy distributions (SEDs) of typical quasars differ markedly from classical theoretical predictions \citep[in which the disc is assumed to be geometrically thin, optically thick, steady  and heated solely by viscous dissipation, e.g., ][]{SS73, Pringle1981}. To a first approximation, the typical quasar SED is flat in $\lambda F_{\lambda}$ over several decades in wavelength \citep{Elvis1994, Richards2006b}. Futhermore, real AGN disks seem to be cooler \cite[e.g., ][]{Lawrence2012} and larger \cite[e.g.,][]{Pooley2007, Morgan2010, Morgan2012, Mosquera2011} than the $\alpha$-disk model predicts. We utilize the $\alpha$-disk model in this paper, noting the model is ad hoc only parameterizing disk viscosity, and does not permit predictions of global changes to the disk \cite{King2012}. 

In this article we present the $z=0.378$ quasar SDSS J110057.70-005304.5 (hereafter J1100-0053).  J1100-0053 was a known quasar which was identified as interesting due to its IR light curve. We have spectral observations for J1100-0053 showing a transition in the blue-continuum into a `dim state' where the rest-frame UV flux is suppressed, and then returning to a blue-continuum sloped quasar.  We present a model that invokes changes at the ISCO to be the triggering event for the change in the accretion disk, which along with the changes in the broad emission lines, explains a major change to the disk interior to 150$r_{g}$ (where $r_{\rm g}$ is the gravitational radius; $r_{\rm g}=\frac{GM}{c^2}$) as well as the IR light curves. Critically, our model makes predictions to the future behaviour of J1100-0053.


\begin{figure}
  \centering
  \includegraphics[width=16.00cm, height=10.00cm, trim=0.0cm 0.0cm 0.0cm 0.0cm, clip]
  {../plots/lc/J110057_lc_20171204v1.png}
  \caption[]{
    Multi-wavelength light curve of J1100-0053, including optical data from LINEAR, CRTS, SDSS, PanSTARRS and DECaLS, and mid-IR data from the WISE satellite.  The three vertical lines illustrate the epochs of the three optical spectra presented in Figure 2.  J1100-0053 was flagged for further study due to the extreme IR fading observed by WISE.  Note that the optical emission has been recovering over the past few years; we predict the IR emission will similarly recover over the next years.}
  \label{fig:J110057_LC_CRTS}
\end{figure}

\begin{figure}
  \centering
  \includegraphics[width=17.00cm, height=12.00cm, trim=0.0cm 0.0cm 0.0cm 0.0cm, clip]
  {../plots/spectra/w1100m0052_sdss_wmodels_20171210.pdf}
  \caption[]{Optical spectra of J1100-0053 obtained on MJD 51908 (blue; SDSS), 55302 (red; BOSS) and 57809 (black; Palomar).  Spectra have been renormalized to maintain a constant \oiii luminosity. Over the past two decades, the UV continuum and broad lines have changed significantly for this quasar.  In particular, the 2nd-epoch BOSS spectrum from 2010 shows the rare occurrence of a temporary collapse of the UV continuum.  Smooth lines show three simple thermal accretion disk models of the continuum.  The solid blue line shows an inflated disk with non-zero torque at the ISCO \cite[e.g.,][]{Sirko_Goodman2003}, while the dashed blue line shows the same model, but with zero torque at the ISCO \cite[i.e., a simple $\alpha$-disk model,][]{SS73}.  Torque at the ISCO heats the inner disk, causing it to puff up and become more UV luminous.  The dotted red line shows a modified zero-torque model where the thermal disk emission interior to $80 r_{\rm g}$ is suppressed by a factor of 10. }
  \label{fig:J110057_spectra}
\end{figure}
%%%%%%%%%%%%%%%%%%%%%%%%%%%%%%%%%%%%%%%%%%%%%%%%%%%%%%%%%%%%% 
%%%%%%%%%%%%%%%%%%%%%%%%%%%%%%%%%%%%%%%%%%%%%%%%%%%%%%%%%%%%% 
%%
%%
%%     SECTION 2   SECTION 2   SECTION 2   SECTION 2   SECTION 2   SECTION 2  
%%     SECTION 2   SECTION 2   SECTION 2   SECTION 2   SECTION 2   SECTION 2  
%%     SECTION 2   SECTION 2   SECTION 2   SECTION 2   SECTION 2   SECTION 2  
%%
%%
%%%%%%%%%%%%%%%%%%%%%%%%%%%%%%%%%%%%%%%%%%%%%%%%%%%%%%%%%%%%%
%%%%%%%%%%%%%%%%%%%%%%%%%%%%%%%%%%%%%%%%%%%%%%%%%%%%%%%%%%%%%
\section{Target Selection and Observations}  
We started by matching the SDSS-III Baryon Oscillation Spectroscopic Survey (BOSS) Data Release 12 Quasar catalog \cite[DR12Q; ][]{Paris2017} to the NEOWISE-R IR data (WISE W1 at 3.4$\mu$m, WISE W2 at 4.6$\mu$m). We found $\sim$200 objects identified by a factor of 2 or more change in the observed WISE W1 and W2 bands over the course of typically three or four years \citep[see][and the Supplemental Material for the detailed NEOWISE-R selection]{Meisner2017b}. Scanning these 200 objects, we also examined the change in optical colour using the SDSS and DECaLS imaging surveys in order to identify changes suggestive of changing-look quasars.  From this inspection, a priority list of $\approx70$ quasar targets was derived and we obtained new optical spectroscopy from the Palomar 5m telescope.  J1100-0053 was one of these 70 objects, which had spectra from both SDSS and BOSS and was thus a priority target.

Figure~\ref{fig:J110057_LC_CRTS} presents the light curve of J1100-0053.  Along with WISE IR data, optical data from the SDSS, Catalina Real-time Transient Survey \citep[CRTS;][]{Drake2009, Mahabal2011}, the Lincoln Near-Earth Asteroid Research \citep[LINEAR; ][]{Sesar2011} program and the Panoramic Survey Telescope and Rapid Response System \citep[PanSTARRS;][]{Kaiser2010, Stubbs2010, Tonry2012, Magnier2013} are available. Figure~\ref{fig:J110057_spectra} shows the three optical spectra of J1100-0053 from the SDSS, BOSS and Palomar observations, taken on MJD 51908 (UT 2000 December 30), 55302 (UT 2010 April 16) and 57809 (UT 2017 February 25), respectively.  The first-epoch SDSS spectrum shows a typical blue quasar, but blue continuum then collapses in the second epoch BOSS spectrum taken 10 years later. The blue continuum then returns in the third epoch spectrum taken another 7 years later, albeit at a diminished level relative to the initial spectrum. The Supplemental Material gives further observational details.


- para.3 = I think here is where we could have a whole paragraph emphasizing how crazy/rare the observations are -- that UV collapse is strange.  looking at the date of the BOSS spectrum in Fig.1, were we incredibly lucky on timing?  possibly introduce Guo here as another example of UV collapse.



%%%%%%%%%%%%%%%%%%%%%%%%%%%%%%%%%%%%%%%%%%%%%%%%%%%%%%%%%%%%%
%%%%%%%%%%%%%%%%%%%%%%%%%%%%%%%%%%%%%%%%%%%%%%%%%%%%%%%%%%%%%
%%
%%   SECTION 3   SECTION 3   SECTION 3   SECTION 3   SECTION 3   SECTION 3  
%%   SECTION 3   SECTION 3   SECTION 3   SECTION 3   SECTION 3   SECTION 3  
%%   SECTION 3   SECTION 3   SECTION 3   SECTION 3   SECTION 3   SECTION 3  
%%
%%%%%%%%%%%%%%%%%%%%%%%%%%%%%%%%%%%%%%%%%%%%%%%%%%%%%%%%%%%%%
%%%%%%%%%%%%%%%%%%%%%%%%%%%%%%%%%%%%%%%%%%%%%%%%%%%%%%%%%%%%%
\begin{figure*}
  %% trim=l b r t 
  \includegraphics[width=15.4cm, height=18.75cm, trim=0.0cm 0.0cm 0.0cm 0.0cm, clip]
  {../plots/models/cartoon_v3pnt1.jpg}
  \centering
  \caption[]{
    Cartoon illustration of our model explaining the unusual spectral evolution of J1100-0053. In 2000, corresponding to the SDSS spectral epoch, the quasar has a standard inflated accretion disk, i.e., where non-zero torque at the ISCO heats the inner radii of the accretion disk, causing it to puff up \citep[e.g.,][]{Zimmerman2005}. Circa 2007, a triggering event occurs that deflates the inner disk, possibly due to a shift in the magnetic field configuration leading to zero torque at the ISCO.  This event leaves some scattering clouds, and causing a cooling front to propogate outwards in the accretion disk, traveling on the $t_{\rm front}$ time-scale. Circa 2012, the cooling front reaches a predicted kink in the accretion disk profile at $\sim$100 $r_{g}$, associated with a shift in the accretion disk opacity \citep[e.g., Figure 2 of ][]{Sirko_Goodman2003}.  The cooling front then reflects back, re-heating the inner accretion disk. We predict that in the next year, the quasar should roughly return to its initial state.}
  \label{fig:J110057_diskmodel}
\end{figure*}
\section{Discussion}   
%- pretty good as is, but maybe that opening paragraph could use a little more glue connecting it to the previous paragraph.  also, you could impinge upon the word count a bit more in this key part of the paper, explaining things a bit more leisurely. also, as i noted, I'm not wild with how we conclude the main part of the paper.

Our model of thermal emission from a multicolour disk implies changes in the region from the ISCO to $\sim$few tens-100 $r_{\rm g}$ are required to suppress flux into the observed $g$-band. In particular, we suggest a physical collapse of the disk scale height due to a cooling front propagating outward from the ISCO.

For J1100-0053 we apply our model as follows. We start with an inflated (slim) disk. We assume a non-zero torque at the ISCO and $h/r\sim0.2$ inside of $r\sim100 r_{\rm g}$. This is the initial state circa 2000 (MJD 51900). A non-zero torque at the ISCO implies that matter in the plunging region is connected (however weakly) to matter outside the ISCO, probably by magnetic fields \cite[e.g., ][]{Gammie1999, Agol_Krolik2000}. A non-zero torque at the ISCO maintains a hotter innermost disk than a condition of zero torque at the ISCO, and an assumption of non-zero torque is particularly appropriate if disk viscosity and accretion are driven by magnetic fields. In order to explain data subsequent to 2007, we assume a cooling front propagates out from the ISCO over a timescale $t_{\rm front}$. The front propagation timescale is $ t_{\rm front}  \sim  10 \; {\rm years} \left(\frac{h/r}{0.05}\right)^{-1}  \left(\frac{\alpha}{0.3}\right)^{-1}   \left(\frac{r}{225r_{g}}\right)^{3/2}  \frac{r_{g}}{c}$, where $h$ and $r$ are the scale-height and radius of the accretion disk respectively, $\alpha$ is the traditional kinematic viscosity
The simplest explanation is that the non-zero torque condition at the ISCO changes to a (more nearly) zero-torque condition, leading to a dramatically cooler, thinner disk near the ISCO. As the cooling front propagates, the drop in temperature leads to a drop in flux. Our model requires cooler regions behind the front to emit 10\% the flux of the initial hotter disk, and the disk height drops by a factor $\sim$2. The dimming of the inner disk causes a drop in the ionizing photon flux ($L_{\rm ion}$), which will cause: the Balmer lines to drop in flux after a light travel time of months and the IR from the outer disk/torus to drop in flux after a light travel time of $\sim$3 years.

If the inner accretion disk is usually inflated \cite[see e.g., ][]{Sirko_Goodman2003, Thompson2005, Hopkins_Quataert2011}, such a cooling front will naturally produce: 1) a collapse in the scale height of the disk; 2) a decrease in flux moving from UV to longer optical wavelengths; 3) a temporarily thicker scattering atmosphere, further decreasing flux at short wavelengths.  This model implies changes to the optical emission moving from shorter to longer wavelengths (as the radius of the cooling front increases), on months-to-years-long timescales. It also predicts a longer time to recover the original flux (compared to the initial collapse) as a front will move more slowly in a thinner disk (see Fig.~2). A decrease in the UV flux would be expected to cause a decrease in IR flux, as the heating of the IR-emitting dusty torus is reduced; however, there should be a delay due to light travel time as well \cite[e.g., ][]{Jun2015}.

Since we assume the disk starts in a puffed-up state ($h/r \sim 0.2$) and since the front cooling time (equation 5 in the Supplemental Material) is inversely proportional to $h/r$ and $\alpha$, the front propagates faster in a puffed-up disk than in a thinner disk. By 2010 (MJD 55300) the front has reached $r\sim50 r_{g}$. During that time, the collapsing disk height increases the number density of scatterers and the temporary cold phase formed during cooling produces the remarkable blue downturn in the 2010 spectrum. The cooling front continues to propagate radially outward but cools less efficiently at larger disk radii. Eventually, a heating front propagates back inwards, analagous to the well-known accretion disk limit cycle mechanism in models of dwarf novae outbursts \cite[e.g., ][]{Cannizzo1998}. The returning heating front travels more slowly because the disk is thinner (and $t_{\rm front}$ is inversely proportional to $h/r$), and will re-inflate the disk as it propagates inwards towards the SMBH. This means the return to normal wll be asymmetric in time, as observed, and the $g$-band bottoms out first, because that wavelength is dominated by emission coming from $r\sim100r_{g}$ (see discussion inthe Supplemental Material).

Using \cite{Ford2018} and \cite{Sirko_Goodman2003}, Figure~\ref{fig:J110057_diskmodel} shows a model for a $M_{\rm BH}=3\times 10^{8} M_{\odot}$, radiative efficiency of $\epsilon=0.1$, accretion rate in units of Eddington accretion, $\dot{M}=0.032$, inner disk radius of $6r_{\rm g}$ and outer disk radius of $10,000 r_{g}$. The resulting model spectra can be seen in Figure~\ref{fig:J110057_diskmodel}.  We expect the front to return to the ISCO in about 2018. That means the broad Balmer lines will come back a few months later, but the WISE IR flux should not come back until about 2021.

\cite{Guo2016} observed a similar event to J1100-0053 with the source SDSS J231742.60+000535.1. However, their object provided an ambiguous case, as the IR brightness of their source did not decline. This is consistent with our model, as their cooling event is relatively brief.  We discuss this object and the \cite{Guo2016} result further in the Supplemental Material. 

In this letter, we have shown that a simple phenomenological model with a propagating cooling front is capable of describing the gross spectral and temporal variations in a changing looking quasar. Our model makes a prediction for this source, testable over the next few years and implies that changing looking quasars as a class are driven by changes near the ISCO, close to the SMBH. By monitoring changing look quasars we introduce new tests of models of accretion disk physics and a new probe of the strong gravity regime.



%%%%%%%%%%%%%%%%%%%%%%%%%%%%%%%%%%%%%%%%%%%%%%%%%%%%%%%%%%%%%
%%%%%%%%%%%%%%%%%%%%%%%%%%%%%%%%%%%%%%%%%%%%%%%%%%%%%%%%%%%%%
%%
%%   SECTION 4   SECTION 4   SECTION 4   SECTION 4   SECTION 4   SECTION 4  
%%   SECTION 4   SECTION 4   SECTION 4   SECTION 4   SECTION 4   SECTION 4  
%%   SECTION 4   SECTION 4   SECTION 4   SECTION 4   SECTION 4   SECTION 4  
%%
%%%%%%%%%%%%%%%%%%%%%%%%%%%%%%%%%%%%%%%%%%%%%%%%%%%%%%%%%%%%%
%%%%%%%%%%%%%%%%%%%%%%%%%%%%%%%%%%%%%%%%%%%%%%%%%%%%%%%%%%%%%
%\section{Method}

%\bibliographystyle{naturemag}
\bibliography{/cos_pc19a_npr/LaTeX/tester_mnras}


\iffalse
\begin{thebibliography}{10}
\expandafter\ifx\csname url\endcsname\relax
  \def\url#1{\texttt{#1}}\fi
\expandafter\ifx\csname urlprefix\endcsname\relax\def\urlprefix{URL }\fi
\providecommand{\bibinfo}[2]{#2}
\providecommand{\eprint}[2][]{\url{#2}}

\bibitem{LaMassa2015}
\bibinfo{author}{{LaMassa}, S.~M.} \emph{et~al.}
\newblock \bibinfo{title}{{The Discovery of the First ``Changing Look'' Quasar:
  New Insights Into the Physics and Phenomenology of Active Galactic Nucleus}}.
\newblock \emph{\bibinfo{journal}{\apj}} \textbf{\bibinfo{volume}{800}},
  \bibinfo{pages}{144} (\bibinfo{year}{2015}).
\newblock \eprint{1412.2136}.

\bibitem{Runnoe2016}
\bibinfo{author}{{Runnoe}, J.~C.} \emph{et~al.}
\newblock \bibinfo{title}{{Now you see it, now you don't: the disappearing
  central engine of the quasar J1011+5442}}.
\newblock \emph{\bibinfo{journal}{\mnras}} \textbf{\bibinfo{volume}{455}},
  \bibinfo{pages}{1691--1701} (\bibinfo{year}{2016}).
\newblock \eprint{1509.03640}.

\bibitem{MacLeod2016}
\bibinfo{author}{{MacLeod}, C.~L.} \emph{et~al.}
\newblock \bibinfo{title}{{A systematic search for changing-look quasars in
  SDSS}}.
\newblock \emph{\bibinfo{journal}{\mnras}} \textbf{\bibinfo{volume}{457}},
  \bibinfo{pages}{389--404} (\bibinfo{year}{2016}).
\newblock \eprint{1509.08393}.

\bibitem{Ruan2016}
\bibinfo{author}{{Ruan}, J.~J.} \emph{et~al.}
\newblock \bibinfo{title}{{Toward an Understanding of Changing-look Quasars: An
  Archival Spectroscopic Search in SDSS}}.
\newblock \emph{\bibinfo{journal}{\apj}} \textbf{\bibinfo{volume}{826}},
  \bibinfo{pages}{188} (\bibinfo{year}{2016}).
\newblock \eprint{1509.03634}.

\bibitem{Yang2017}
\bibinfo{author}{{Yang}, Q.} \emph{et~al.}
\newblock \bibinfo{title}{{Discovery of 21 New Changing-look AGNs in Northern
  Sky}}.
\newblock \emph{\bibinfo{journal}{ArXiv e-prints}}  (\bibinfo{year}{2017}).
\newblock \eprint{1711.08122v1}.

\bibitem{Hutsemekers2017}
\bibinfo{author}{{Hutsem{\'e}kers}, D.}, \bibinfo{author}{{Ag{\'{\i}}s
  Gonz{\'a}lez}, B.}, \bibinfo{author}{{Sluse}, D.}, \bibinfo{author}{{Ramos
  Almeida}, C.} \& \bibinfo{author}{{Acosta Pulido}, J.-A.}
\newblock \bibinfo{title}{{Polarization of the changing-look quasar
  J1011+5442}}.
\newblock \emph{\bibinfo{journal}{\aap}} \textbf{\bibinfo{volume}{604}},
  \bibinfo{pages}{L3} (\bibinfo{year}{2017}).
\newblock \eprint{1707.05540}.

\bibitem{Sheng2017}
\bibinfo{author}{{Sheng}, Z.} \emph{et~al.}
\newblock \bibinfo{title}{{Mid-infrared Variability of Changing-look AGNs}}.
\newblock \emph{\bibinfo{journal}{\apjl}} \textbf{\bibinfo{volume}{846}},
  \bibinfo{pages}{L7} (\bibinfo{year}{2017}).
\newblock \eprint{1707.02686}.

\bibitem{Gezari2017}
\bibinfo{author}{{Gezari}, S.} \emph{et~al.}
\newblock \bibinfo{title}{{iPTF Discovery of the Rapid ``Turn-on'' of a
  Luminous Quasar}}.
\newblock \emph{\bibinfo{journal}{\apj}} \textbf{\bibinfo{volume}{835}},
  \bibinfo{pages}{144} (\bibinfo{year}{2017}).
\newblock \eprint{1612.04830}.

\bibitem{Rumbaugh2017}
\bibinfo{author}{{Rumbaugh}, N.} \emph{et~al.}
\newblock \bibinfo{title}{{Extreme variability quasars from the Sloan Digital
  Sky Survey and the Dark Energy Survey}}.
\newblock \emph{\bibinfo{journal}{ArXiv e-prints}}  (\bibinfo{year}{2017}).
\newblock \eprint{1706.07875}.

\bibitem{SS73}
\bibinfo{author}{{Shakura}, N.~I.} \& \bibinfo{author}{{Sunyaev}, R.~A.}
\newblock \bibinfo{title}{{Black holes in binary systems. Observational
  appearance.}}
\newblock \emph{\bibinfo{journal}{\aap}} \textbf{\bibinfo{volume}{24}},
  \bibinfo{pages}{337} (\bibinfo{year}{1973}).

\bibitem{King2012}
\bibinfo{author}{{King}, A.}
\newblock \bibinfo{title}{{Accretion disc theory since Shakura and Sunyaev}}.
\newblock \emph{\bibinfo{journal}{\memsai}} \textbf{\bibinfo{volume}{83}},
  \bibinfo{pages}{466} (\bibinfo{year}{2012}).
\newblock \eprint{1201.2060}.

\bibitem{Lawrence2012}
\bibinfo{author}{{Lawrence}, A.}
\newblock \bibinfo{title}{{The UV peak in active galactic nuclei: a false
  continuum from blurred reflection?}}
\newblock \emph{\bibinfo{journal}{\mnras}} \textbf{\bibinfo{volume}{423}},
  \bibinfo{pages}{451--463} (\bibinfo{year}{2012}).
\newblock \eprint{1110.0854}.

\bibitem{Pooley2007}
\bibinfo{author}{{Pooley}, D.}, \bibinfo{author}{{Blackburne}, J.~A.},
  \bibinfo{author}{{Rappaport}, S.} \& \bibinfo{author}{{Schechter}, P.~L.}
\newblock \bibinfo{title}{{X-Ray and Optical Flux Ratio Anomalies in Quadruply
  Lensed Quasars. I. Zooming in on Quasar Emission Regions}}.
\newblock \emph{\bibinfo{journal}{\apj}} \textbf{\bibinfo{volume}{661}},
  \bibinfo{pages}{19--29} (\bibinfo{year}{2007}).
\newblock \eprint{astro-ph/0607655}.

\bibitem{Morgan2010}
\bibinfo{author}{{Morgan}, C.~W.}, \bibinfo{author}{{Kochanek}, C.~S.},
  \bibinfo{author}{{Morgan}, N.~D.} \& \bibinfo{author}{{Falco}, E.~E.}
\newblock \bibinfo{title}{{The Quasar Accretion Disk Size-Black Hole Mass
  Relation}}.
\newblock \emph{\bibinfo{journal}{\apj}} \textbf{\bibinfo{volume}{712}},
  \bibinfo{pages}{1129--1136} (\bibinfo{year}{2010}).
\newblock \eprint{1002.4160}.

\bibitem{Morgan2012}
\bibinfo{author}{{Morgan}, C.~W.} \emph{et~al.}
\newblock \bibinfo{title}{{Further Evidence that Quasar X-Ray Emitting Regions
  are Compact: X-Ray and Optical Microlensing in the Lensed Quasar Q
  J0158-4325}}.
\newblock \emph{\bibinfo{journal}{\apj}} \textbf{\bibinfo{volume}{756}},
  \bibinfo{pages}{52} (\bibinfo{year}{2012}).
\newblock \eprint{1205.4727}.

\bibitem{Mosquera2011}
\bibinfo{author}{{Mosquera}, A.~M.} \& \bibinfo{author}{{Kochanek}, C.~S.}
\newblock \bibinfo{title}{{The Microlensing Properties of a Sample of 87 Lensed
  Quasars}}.
\newblock \emph{\bibinfo{journal}{\apj}} \textbf{\bibinfo{volume}{738}},
  \bibinfo{pages}{96} (\bibinfo{year}{2011}).
\newblock \eprint{1104.2356}.

\bibitem{Balbus_Hawley1991}
\bibinfo{author}{{Balbus}, S.~A.} \& \bibinfo{author}{{Hawley}, J.~F.}
\newblock \bibinfo{title}{{A powerful local shear instability in weakly
  magnetized disks. I - Linear analysis. II - Nonlinear evolution}}.
\newblock \emph{\bibinfo{journal}{\apj}} \textbf{\bibinfo{volume}{376}},
  \bibinfo{pages}{214--233} (\bibinfo{year}{1991}).

\bibitem{McKernan2014}
\bibinfo{author}{{McKernan}, B.}, \bibinfo{author}{{Ford}, K.~E.~S.},
  \bibinfo{author}{{Kocsis}, B.}, \bibinfo{author}{{Lyra}, W.} \&
  \bibinfo{author}{{Winter}, L.~M.}
\newblock \bibinfo{title}{{Intermediate-mass black holes in AGN discs - II.
  Model predictions and observational constraints}}.
\newblock \emph{\bibinfo{journal}{\mnras}} \textbf{\bibinfo{volume}{441}},
  \bibinfo{pages}{900--909} (\bibinfo{year}{2014}).
\newblock \eprint{1403.6433}.

\bibitem{Mainzer2014}
\bibinfo{author}{{Mainzer}, A.} \emph{et~al.}
\newblock \bibinfo{title}{{Initial Performance of the NEOWISE Reactivation
  Mission}}.
\newblock \emph{\bibinfo{journal}{\apj}} \textbf{\bibinfo{volume}{792}},
  \bibinfo{pages}{30} (\bibinfo{year}{2014}).
\newblock \eprint{1406.6025}.

\bibitem{Meisner2017a}
\bibinfo{author}{{Meisner}, A.~M.}, \bibinfo{author}{{Lang}, D.} \&
  \bibinfo{author}{{Schlegel}, D.~J.}
\newblock \bibinfo{title}{{Deep Full-sky Coadds from Three Years of WISE and
  NEOWISE Observations}}.
\newblock \emph{\bibinfo{journal}{\aj}} \textbf{\bibinfo{volume}{154}},
  \bibinfo{pages}{161} (\bibinfo{year}{2017}).
\newblock \eprint{1705.06746}.

\bibitem{Meisner2017b}
\bibinfo{author}{{Meisner}, A.~M.} \emph{et~al.}
\newblock \bibinfo{title}{{Searching for Planet Nine with Coadded WISE and
  NEOWISE-Reactivation Images}}.
\newblock \emph{\bibinfo{journal}{\aj}} \textbf{\bibinfo{volume}{153}},
  \bibinfo{pages}{65} (\bibinfo{year}{2017}).
\newblock \eprint{1611.00015}.

\bibitem{Assef2017}
\bibinfo{author}{{Assef}, R.~J.} \emph{et~al.}
\newblock \bibinfo{title}{{The WISE AGN Catalog}}.
\newblock \emph{\bibinfo{journal}{1706.09901v1}}  (\bibinfo{year}{2017}).
\newblock \eprint{1706.09901}.

\bibitem{Paris2017}
\bibinfo{author}{{P{\^a}ris}, I.}, \bibinfo{author}{{Petitjean}, P.},
  \bibinfo{author}{{Ross}, N.~P.} \emph{et~al.}
\newblock \bibinfo{title}{{The Sloan Digital Sky Survey Quasar Catalog: Twelfth
  data release}}.
\newblock \emph{\bibinfo{journal}{\aap}} \textbf{\bibinfo{volume}{597}},
  \bibinfo{pages}{A79} (\bibinfo{year}{2017}).
\newblock \eprint{1608.06483}.

\bibitem{Drake2009}
\bibinfo{author}{{Drake}, A.~J.} \emph{et~al.}
\newblock \bibinfo{title}{{First Results from the Catalina Real-Time Transient
  Survey}}.
\newblock \emph{\bibinfo{journal}{\apj}} \textbf{\bibinfo{volume}{696}},
  \bibinfo{pages}{870--884} (\bibinfo{year}{2009}).
\newblock \eprint{0809.1394}.

\bibitem{Mahabal2011}
\bibinfo{author}{{Mahabal}, A.~A.} \emph{et~al.}
\newblock \bibinfo{title}{{Discovery, classification, and scientific
  exploration of transient events from the Catalina Real-time Transient
  Survey}}.
\newblock \emph{\bibinfo{journal}{Bulletin of the Astronomical Society of
  India}} \textbf{\bibinfo{volume}{39}}, \bibinfo{pages}{387--408}
  (\bibinfo{year}{2011}).
\newblock \eprint{1111.0313}.

\bibitem{Sesar2011}
\bibinfo{author}{{Sesar}, B.} \emph{et~al.}
\newblock \bibinfo{title}{{Exploring the Variable Sky with LINEAR. I.
  Photometric Recalibration with the Sloan Digital Sky Survey}}.
\newblock \emph{\bibinfo{journal}{\aj}} \textbf{\bibinfo{volume}{142}},
  \bibinfo{pages}{190} (\bibinfo{year}{2011}).
\newblock \eprint{1109.5227}.

\bibitem{Kaiser2010}
\bibinfo{author}{{Kaiser}, N.} \emph{et~al.}
\newblock \bibinfo{title}{{The Pan-STARRS wide-field optical/NIR imaging
  survey}}.
\newblock In \emph{\bibinfo{booktitle}{Society of Photo-Optical Instrumentation
  Engineers (SPIE) Conference Series}}, vol. \bibinfo{volume}{7733} of
  \emph{\bibinfo{series}{Society of Photo-Optical Instrumentation Engineers
  (SPIE) Conference Series}}, \bibinfo{pages}{0} (\bibinfo{year}{2010}).

\bibitem{Stubbs2010}
\bibinfo{author}{{Stubbs}, C.~W.} \emph{et~al.}
\newblock \bibinfo{title}{{Precise Throughput Determination of the PanSTARRS
  Telescope and the Gigapixel Imager Using a Calibrated Silicon Photodiode and
  a Tunable Laser: Initial Results}}.
\newblock \emph{\bibinfo{journal}{\apjs}} \textbf{\bibinfo{volume}{191}},
  \bibinfo{pages}{376--388} (\bibinfo{year}{2010}).
\newblock \eprint{1003.3465}.

\bibitem{Tonry2012}
\bibinfo{author}{{Tonry}, J.~L.} \emph{et~al.}
\newblock \bibinfo{title}{{The Pan-STARRS1 Photometric System}}.
\newblock \emph{\bibinfo{journal}{\apj}} \textbf{\bibinfo{volume}{750}},
  \bibinfo{pages}{99} (\bibinfo{year}{2012}).
\newblock \eprint{1203.0297}.

\bibitem{Magnier2013}
\bibinfo{author}{{Magnier}, E.~A.} \emph{et~al.}
\newblock \bibinfo{title}{{The Pan-STARRS 1 Photometric Reference Ladder,
  Release 12.01}}.
\newblock \emph{\bibinfo{journal}{\apjs}} \textbf{\bibinfo{volume}{205}},
  \bibinfo{pages}{20} (\bibinfo{year}{2013}).
\newblock \eprint{1303.3634}.

\bibitem{Gammie1999}
\bibinfo{author}{{Gammie}, C.~F.}
\newblock \bibinfo{title}{{Efficiency of Magnetized Thin Accretion Disks in the
  Kerr Metric}}.
\newblock \emph{\bibinfo{journal}{\apjl}} \textbf{\bibinfo{volume}{522}},
  \bibinfo{pages}{L57--L60} (\bibinfo{year}{1999}).
\newblock \eprint{astro-ph/9906223}.

\bibitem{Agol_Krolik2000}
\bibinfo{author}{{Agol}, E.} \& \bibinfo{author}{{Krolik}, J.~H.}
\newblock \bibinfo{title}{{Magnetic Stress at the Marginally Stable Orbit:
  Altered Disk Structure, Radiation, and Black Hole Spin Evolution}}.
\newblock \emph{\bibinfo{journal}{\apj}} \textbf{\bibinfo{volume}{528}},
  \bibinfo{pages}{161--170} (\bibinfo{year}{2000}).
\newblock \eprint{astro-ph/9908049}.

\bibitem{Sirko_Goodman2003}
\bibinfo{author}{{Sirko}, E.} \& \bibinfo{author}{{Goodman}, J.}
\newblock \bibinfo{title}{{Spectral energy distributions of marginally
  self-gravitating quasi-stellar object discs}}.
\newblock \emph{\bibinfo{journal}{\mnras}} \textbf{\bibinfo{volume}{341}},
  \bibinfo{pages}{501--508} (\bibinfo{year}{2003}).
\newblock \eprint{astro-ph/0209469}.

\bibitem{Thompson2005}
\bibinfo{author}{{Thompson}, T.~A.}, \bibinfo{author}{{Quataert}, E.} \&
  \bibinfo{author}{{Murray}, N.}
\newblock \bibinfo{title}{{Radiation Pressure-supported Starburst Disks and
  Active Galactic Nucleus Fueling}}.
\newblock \emph{\bibinfo{journal}{\apj}} \textbf{\bibinfo{volume}{630}},
  \bibinfo{pages}{167--185} (\bibinfo{year}{2005}).
\newblock \eprint{astro-ph/0503027}.

\bibitem{Hopkins_Quataert2011}
\bibinfo{author}{{Hopkins}, P.~F.} \& \bibinfo{author}{{Quataert}, E.}
\newblock \bibinfo{title}{{An analytic model of angular momentum transport by
  gravitational torques: from galaxies to massive black holes}}.
\newblock \emph{\bibinfo{journal}{\mnras}} \textbf{\bibinfo{volume}{415}},
  \bibinfo{pages}{1027--1050} (\bibinfo{year}{2011}).
\newblock \eprint{1007.2647}.

\bibitem{Jun2015}
\bibinfo{author}{{Jun}, H.~D.} \emph{et~al.}
\newblock \bibinfo{title}{{Infrared Time Lags for the Periodic Quasar PG
  1302-102}}.
\newblock \emph{\bibinfo{journal}{\apjl}} \textbf{\bibinfo{volume}{814}},
  \bibinfo{pages}{L12} (\bibinfo{year}{2015}).
\newblock \eprint{1511.01515}.

\bibitem{Cannizzo1998}
\bibinfo{author}{{Cannizzo}, J.~K.}
\newblock \bibinfo{title}{{On the M$_{V}$(peak) versus Orbital Period Relation
  for Dwarf Nova Outbursts}}.
\newblock \emph{\bibinfo{journal}{\apj}} \textbf{\bibinfo{volume}{493}},
  \bibinfo{pages}{426--430} (\bibinfo{year}{1998}).
\newblock \eprint{astro-ph/9712210}.

\bibitem{Ford2018}
\bibinfo{author}{{Ford}, K.~E.~S.} \emph{et~al.}
\newblock \emph{\bibinfo{journal}{in prep.}}  (\bibinfo{year}{2018}).

\bibitem{Guo2016}
\bibinfo{author}{{Guo}, H.} \emph{et~al.}
\newblock \bibinfo{title}{{The Optical Variability of SDSS Quasars from
  Multi-epoch Spectroscopy. III. A Sudden UV Cutoff in Quasar SDSS
  J2317+0005}}.
\newblock \emph{\bibinfo{journal}{\apj}} \textbf{\bibinfo{volume}{826}},
  \bibinfo{pages}{186} (\bibinfo{year}{2016}).
\newblock \eprint{1605.07301}.

\end{thebibliography}

\fi


\end{document}
