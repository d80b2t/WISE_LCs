\documentclass[11pt,a4paper]{article}
\input{format}

\setlength {\textwidth}{160mm} 
\setlength {\textheight}{255mm}
\topmargin=-15.00mm
\oddsidemargin=-5.0mm

\begin{document}

   \title{Supplemental Material for J110057}
\maketitle
%%Saavik comments after %sf

\section*{Further Observational Details}

\subsection*{Selection in SDSS-III BOSS of J110057}
SDSS J110057.71-005304.5 was first detected by the ROSAT and appears
in the All-Sky Survey Bright Source Catalogue \citep[RASS-BSC;
][]{Appenzeller1998, Voges1999}.  J110057 was then imaged by the Sloan
Digital Sky Survey (SDSS) and satisfied a number of spectroscopic
targeting flags\footnote{SERENDIP\_BLUE, ROSAT\_D, ROSAT\_C, ROSAT\_B,
QSO\_SKIRT, ROSAT\_A, see \citet{EDR} and \citet{Richards2002} for
flag descriptions.}  making it a quasar target. A spectrum was obtain
on MJD 51908 (Plate 277, Fiber 212) and the spectrum of a $z=0.378$
quasar was catalogued in the SDSS Early Data Release
\citep{Stoughton2002, Schneider2002}. The physical properties of
J110057, derived from the MJD 51908 spectrum and using the methods in
\citet{Shen2011}, are given in Table~\ref{tab:Shen_props}.

\begin{table}[]
    \centering
    \begin{tabular}{l l }
      \hline \hline 
      Quantity                                         &  Value \\
      \hline 
      SDSS name                                     &   J110057.71-005304.5 \\
      R.A. / deg                &  165.240463 \\
      Declination / deg    &   -0.884586 \\ 
      redshift, $z$                                    &   0.3778$\pm$0.0003  \\
      $M_{i}(z=2)$  / mag                          &   -24.48  \\
      log $(L_{\rm bol} / {\rm erg s}^{-1}) $  &  45.78$\pm$0.02 \\
      log $(M_{\rm BH} / M_{\odot})  $           &  8.83$\pm$0.14 \\
      Eddington ratio                                &        0.070 \\
      \hline \hline 
    \end{tabular}
    \caption{Physical properties of J110057 using the methods from 
      \citet{Shen2011}.} 
    \label{tab:Shen_props}
\end{table}

The second epoch spectrum is from the SDSS-III Baryon Oscillation
Spectroscopic Survey \citep[BOSS; ][]{Dawson2013} and shows the
downturn at $\lesssim$4300\AA\ .  SDSS-III BOSS actively vetoed
low-$z$ QSOs \citep{Ross2012}, and it was due to J110057 being
selected as an ancillary target via a white dwarf program
\citep{Kepler2015, Kepler2016} that a second spectral epoch was
obtained.  Since J110057 was not a BOSS QSO target, it is not subject
to the ``blue offset'', see \citet{Margala2016}.

A third epoch spectrum was obtained from the Palomar Hale 5m telescope
using the DBSP instrument.  Two exposures of 600s+300s were taken in
good conditions. Features to note include the continuum straddling
\mgii being blue in the 2017 spectrum, as it was for the SDSS spectrum
in 2000, as opposed to red, as it was for the BOSS spectrum in 2010.

J110057 is in Data Release 3 (DR3) of the Dark Energy 
Camera Legacy Survey (DECaLS), where there are 8, 3 and 9 exposures in
the $g$, $r$ and $z$-band respectively. The $g$- and $r$-band
observations are separated by roughly a year, ($56707 \leq g_{\rm MJD}
\leq 56727$ and $56367 \leq r_{\rm MJD} \leq 56367$). The $z$-band
observations span almost 3 years ($56383 \leq z_{\rm MJD} \leq
57398)$.


\subsection*{Selection in NEOWISE-R of J110057}
%%  This is text from Aaron's email from 15th Feb, 2017, sent to Daniel, 
%%  for the WISE J1052+1519 outline/paper

WISE W1 and W2 lightcurves for $\sim$200,000 SDSS spectroscopic
quasars were obtained. These light curves span from the beginning of
the WISE mission (2010 January) through the first-year of NEOWISE-R
operations (2014 December). The W1/W2 light curves are obtained by
performing forced photometry at the locations of DECam-detected
optical sources \citep{Lang2014, Meisner2017a, Meisner2017b}.  This
forced photometry is performed on time-resolved unWISE coadds
\citep{Lang2014}, each of which represents a stack of $\sim$12
exposures. A given sky location is observed by WISE for $\sim$1 day
once every six months, which means that the forced photometry light
curves typically have four coadd epochs available. Coadd epochs of a
given object are separated by a minimum of six months and a maximum of
four years. The coaddition removes the possibility of probing
variability on $\lesssim$1 day time scales, but pushes $\approx$1.4
magnitudes deeper than individual exposures while removing virtually
all single-exposure artifacts (e.g. cosmic rays and satellites).

Approximately $\sim$30,000\footnote{NPR NOTE:: This number seems low to
me; NPR to double check; can also triple check with Aaron} of the
SDSS/BOSS quasars with W1/W2 light-curves available are
``IR-bright'', in that they are above both the W1 and W2
single exposure thresholds and therefore detected at very high
significance in the coadds. For this ensemble of objects, the typical
variation in each quasar's measured (W1-W2) color is 0.06 magnitudes.
This includes statistical and systematic errors which are expected to
contribute variations at the few hundredths of a magnitude level. The
typical measured single-band scatter is 0.07 magnitudes in each of W1
and W2.

We undertook a search for outliers relative to these trends.
Specifically, we selected objects with the following characteristics:
\begin{itemize}
    \item Monotonic variation in both W1 and W2.
    \item W1 versus W2 flux correlation coefficient $\geq0.9$.
    \item $>0.5$ mag peak-to-peak variation in either W1 or W2.
\end{itemize}
This yields a sample of 248 sources. 31 of these are assumed to be
blazars due to the presence of FIRST radio counterparts, and we
discount them in the further analyses here. Another 22 are outside of
the FIRST footprint, leaving 195 quasars in our IR-variable sample.

A link to our sample can be found here:
\href{http://portal.nersc.gov/project/cosmo/temp/ameisner/qso\_pages\_v01/}
{\tt qso\_pages\_v01} and links to the catalogs are given here:
\href{http://portal.nersc.gov/project/cosmo/temp/ameisner/dr3_wise_lc_sample.fits.gz}{{\tt
dr3\_wise\_lc\_sample.fits.gz}} and here:
\href{http://portal.nersc.gov/project/cosmo/temp/ameisner/dr3_wise_lc_metrics_all_qso.fits.gz}{{\tt
dr3\_wise\_lc\_sample.fits.gz}}.  The first catalog has 248 rows,
which are the highly IR-variable sample of objects.  The second
catalog is the full \hbox{200 622} quasar sample quasars that have
``good'' WISE light curves available in DECaLS DR3. In each file,
there are 3 extensions: the first extension are the WISE light curve
summary metrics; the second extension the DECaLS DR3 data for each
object, and the third extension, the SDSS data for each object.  A full
characterization of the typical mid-IR quasar variability will be
presented separately.

\subsection*{Additional Multiwavelength data for J110057}
Checking the data archives we found there was no source within 30
arcsec in the VLA FIRST, i.e., at 21 cm radio frequencies for J11057.
None of the {\it Hubble Space Telescope}, the {\it Spitzer Space
Telescope} or the {\it Kepler} Mission has observed J110057.  It is
also not in the Hyper Suprime-Cam (HSC)
\href{https://hsc-release.mtk.nao.ac.jp/doc/}{Data Release 1}
\citep{Aihara2017} footprint. There is the detection in ROSAT (which
triggers using the 2nd all-sky survey (2RXS; Boller et al. 2016, A\&A,
588, 103) as 2RXS J110058.1-005259 with 27.00 counts (count error
6.14) and a count rate = 0.06$\pm$0.01. The NED gives J110057 as
$1.27\pm0.28 \times^{-12}$ erg/cm$^{2}$/s in the 0.1-2.4 keV range
(unabsorbed flux). J110057 is not in either {\it Chandra} or {\it
XMM-Newton}.


%\clearpage
\section*{Further Model Details}
In this section we discuss several models trying to explain the light
curve and spectral behaviour of J110057. Ultimately, we are forced
towards a model that combines a cooling front propagating in the
accretion disk along with changes in the disk opacity.

\subsection*{Scenario I: Obscuring Cloud model}
We first explore the possibility that an obscuring cloud, or clouds,
cause the observed light curve and spectral behaviour of J110057.  In
this scenario, one would require the obscuring cloud(s) to cross the
line of sight. In order to explain both the IR drop and broadline
disappearance, one would also need the cloud(s) to block most of the
inner disk such that the ionizing radiation could not impact on the
BLR
%sf added BLR for clarity--too many 'clouds'
clouds or the torus for a period of months-years.  Another requirement
is an explanation of why the light curves `recover' after a period of
$\sim 2500$ days (observed-frame); i.e., the light curves do not
rapidly return to their original flux levels once the obscuring event
is over.

Clouds should not typically infall; they need to lose angular momentum
if they are drawn from a distribution with Keplerian orbits, and even
if they do lose angular momentum, e.g., in a collision with
approximately equal mass, they would likely be either destroyed or no
longer coherent. Further issues 
%sf typo arrise
arise, since the freefall timescales
are,
\begin{equation}
    t_{{\rm ff}}   \sim 100   \rm{yr}  \left(\frac{r}{0.4\rm{pc}}\right)^{3/2} 
                                            \left(\frac{M}{10^{8}M_{\odot}}\right)^{-1}
\end{equation}
and Kelvin-Helmholtz instabilities would destroy the clouds within the
cloud-crushing time, \citep[e.g., ][]{Nagakura2008, Hopkins2013,
Shiokawa2015, Bae2016},
%sf removed duplicate: the cloud-crushing time
given by
\begin{equation}
    t_{\rm cc} \sim 100\rm{yr} \left(\frac{\rho_{cloud}/\rho_{medium}}{10^{6}}\right)^{1/2} 
                                            \left(\frac{R_{\rm cloud}}{4 \times 10^{10}\rm{km}}\right) 
                                            \left(\frac{v_{\rm rel}}{10^{4}\rm{km/s}}\right)^{-1}
\end{equation}
Thus, even if clouds did infall, they would end up fragmented, which
should pollute the inner disk (see below for this discussion applied
to the circumstances in \citet{Guo2016}).

The dust in the cloud is then well inside the dust sublimation radius
\begin{equation}
    R_{\rm dust} \approx 0.4\rm{pc}\left(\frac{L}{10^{45}\rm{erg/s}}\right)^{1/2}
                                                   \left(\frac{T_{\rm sub}}{1500\rm{K}}\right)^{2.6}
\end{equation}
and so the dust will be destroyed in the $\sim$100 year free-fall from
the dust-sublimation region. Hence, one can not absorb the UV spectrum
with dust, since it will have been sublimated well before it arrives
at the inner disk.

Typical extinction profiles from clouds with hydrogen column densities
of $N_{H} \sim 10^{21}-10^{22} \rm{cm}^{-2}$ (comparable to the range
expected for NLR-BLR cloud densities) are linear in 
1/$\lambda$ with the 2175 \AA\ feature, \citep[e.g., Figure 4
of][]{Gordon2003}, and not at all like the asymptotic drop off at
$1/\lambda=3 (\mu m^{-1})=1/300\rm{nm}$ in \citet{Guo2016} or in our
2010 spectrum. Note that in the extinction profiles in
\citet{Gordon2003}, there is a local maximum near $4.5 \sim 1/\lambda (\mu
m^{-1})$, implying $\lambda \sim 0.2\mu$m in these
cloud extinction profiles. %sf: adding weasel words and clarifying details
This could correspond to broadened Ly$\alpha$ absorption;
if this is broadened in a turbulent environment and combined with
strong oxygen and carbon edges in a colder phase medium, it is possible to
generate the falling off at 1/(200-300nm) in our 2010 spectrum (and
Guo et al.'s spectrum). With all these considerations, we make a strong case
that the behaviour observed in J110057 \emph{cannot} be extinction due
to a dusty cloud.


\subsection*{Scenario II: Accretion Disk model}
Having discounted an obscuring event as the explanation for J110057,
we turn to an accretion disk model.

%sf: rewriting para below
The early 2000s spectrum is well fit with a thin, \citet{SS73}
$\alpha$-disk. The 2010 spectrum and the sharp fall-off
at $\sim 200-300$nm, is not reproducible using a different temperature
profile alone, even one where the entire inner disk (unphysically)
vanishes. This is due to the width of the Planck function in
wavelength space.
For the same reasons, a gray absorber
model with uniformly suppressed emission at small disk radii is also
incapable of fitting our 2010 (or Guo et al.'s) spectrum. 
Wavelength dependent absorption, combined with a lower
disk emissivity is required. 

\smallskip \smallskip
\noindent
\textbf{\textsc{Model `A': Switching states to an ADAF: }}
%sf: more weasel/clarifying words
The broadband spectrum of NGC 1097 from \citet{Nemmen2006} initially appears similar
to the UV/optical 2010 spectrum of J110057.  In \citet[][e.g., their
Figure 4]{Nemmen2006}, there are disk model components that look
similar to the fall-off at 200nm observed in the J11057 2010 spectrum.
This would involve a thin disk component extending from $\sim
450r_{g}$ to the outer regions of the disk. Figure 4 in
\citet{Nemmen2006} shows the Multicolor Disk (MCD) blackbody-like
model component from the thin disk at $R>450r_{g}$ (their long dashed
line) dramatically decreasing at $\sim 10^{15}$Hz ($\sim 300$nm).
\citet{Nemmen2006} model the disk region interior to this as 
an advection-dominated accretion flow (ADAF), at a power (in $\nu
L_{\nu}$), an order of magnitude lower than the MCD in the optical,
but spanning from the X-ray to the far-IR.\footnote{A change to 
a radiatively inefficient accretion flow (RIAF) is also possible in 
this model.}


\begin{figure*}
  \centering
  %% trim=l b r t 
  \includegraphics[width=11.00cm, height=8.0cm, trim=0.3cm 0.0cm 2.0cm 0.0cm, clip]
  {../plots/models/mcd_gap_v3_1_b1.png}
  \caption[]{
    J110057 data (cyan line 2000 spectrum; magenta line 2010 spectrum) and
    three models.  Temperature suppression inside $r_{\rm alt}$ such that
    the spectral flux is down (from ZT) by factor f$_{\rm dep}$ (SUP): T$_{\rm SUP}$
  }
  \label{fig:disk_suppression}
\end{figure*}
Can J110057 switch states from a thin disk quasar to an ADAF at small
radii with the thin disk surviving at large radii?  Assuming the
transition happens due to an instability on the thermal timescale of
the disk, then at large radii the thermal timescale is
\begin{equation}
    t_{\rm th} \sim 14 \; {\rm years} \; \left(\frac{\alpha}{0.03}\right)^{-1}
                                                \left(\frac{R}{450r_{g}}\right)^{3/2} 
                                                        \frac{r_{g}}{c}
\end{equation} 
and is too long given the observations. However, if the viscosity
parameter $\alpha$ increases to $\alpha \approx 0.3$, as suggested by
\citet{King2007}, then the thermal timescale is $t_{\rm th} \sim 1.4$
year and the front timescale is
\begin{equation}
    t_{\rm front}  \sim  10{\rm yr}  \left(\frac{h/R}{0.05}\right)^{-1}
                                                 \left(\frac{\alpha}{0.3}\right)^{-1}  
                                                 \left(\frac{R}{450r_{g}}\right)^{3/2}  
                                                         \frac{r_{g}}{c}
\label{eqn:t_front}
\end{equation}
which is plausible, if there exists a very viscous disk and the effect
propagates outwards on a timescale of $\leq 10$ years from the inner
disk. This would suppress the UV/X-ray emission from the RIAF (down by
a few orders of magnitude from the intensity expected from a thin disk
intensity) and explain the broadline behaviour.  ADAF spectra are flat
in $\nu L_{\nu}$ \citet{Narayan1998, Abramowicz2002, Abramowicz2013},
and convective ADAFs rise towards X-ray energies. ADAFs exist at lower
luminosity, where $\epsilon \sim 0.005$ for $L=\epsilon \dot{M}
c^{2}$, lower than the fiducial $\epsilon \sim 0.1$ for a classic thin
disk luminosity.

%sf: many edits in 2 para below
However, suppressing the flux from the inner disk radii ($\lesssim 450 r_{g}$)
in the low temperature thin disk model \citep{Narayan1997, Gammie1999,
Agol_Krolik2000, Afshordi_Paczynski2003, Ford2018}, by a factor of
$20$ would still not describe the 2010 spectrum. To restore the thin disk
spectrum by 2016, the disk change has to propagate back inwards, most
of the way to the ISCO and therefore $t_{\rm front}$ needs to be
shorter. This requires $h/R$ to be larger in
Equation~\label{eqn:t_front} above, by a factor of $\sim 2$.

It is unclear what physical processes would trigger the change of
state to an ADAF and then cool back down to a thin disk. However, more
of an issue is that supressing the MCD temperature profile inside a
radius of $R_{alt} = 450 r_{g}$ leads to a collapse in the total
flux compared to unperturbed disk. We show some example cases
in~\ref{fig:disk_suppression}. Clearly, these scenarios are difficult
to reconcile with our data.

\smallskip \smallskip
\noindent
\textbf{\textsc{Model `B': Propagating of a Cooling Front: }}
An alternative model connected to the accretion disk is that a
\emph{cooling} front propagates through the thin disk.  In order to
reproduce the steep fall at $\lambda \leq 200$nm in the 2010 spectrum,
a cool phase leads to absorption at short wavelengths.

Initially a modestly fat disk ($h/R \sim 0.2$) with a modest $\alpha$,
cools from the ISCO and propagates outward in a cooling front,
collapsing the disk. As the hot disk ($\sim 10^{5-6}$K) cools, it
fragments into cooler clumps around $\sim 10^{4}$K \citep[see e.g.,
][]{McCourt2016}.  The main coolants are resonance lines in carbon and
oxygen \citep[see e.g., Fig. 18 in ][]{Sutherland_Dopita1993} The
ionization energies for carbon and carbon are 11.26 and 13.61 eV,
respectively, i.e., $\sim 100$nm, and hence at wavelengths $<100$nm
the disk opacity will increase dramatically in an edge. %sf: edits below
However, the
gas in the disk is both pressure, turbulent and Doppler broadened, so
these ionization edges will manifest around 100nm with decreasing
opacity to shorter wavelengths as
\begin{equation}
  \kappa \propto \rho T^{-1/2} \nu^{-3}
\end{equation}
for Kramers' opacities. This implies $\kappa \propto \lambda^{3}$
at increasing wavelengths up to the ionization edge around $100$nm.
These features will be blurred (by the broadening) and the ionization
edges due to the C and O resonance lines in the cool phase of this
disk will be span $50-200$nm, depressing the flux at these energies.

The 2010 spectrum in this model comes from a cooler disk plus the
increased opacity at short wavelengths in the cooler phase. Heating
occurs from the outside in, explaining the 2016 spectrum and
asymmetric recovery in photometry.  Since the
optical continuum has been rising again since mid-2016, this leads to
a prediction of a rise in hydrogen emission line flux in the next few
months (2018). The infrared flux returns in 2021. 

%Right now, these are the only possible disk models that I can think of that can explain the 2010 and 2016 spectra and the timescales.\\


\section*{Comparison with SDSS J2317+0005 from Guo et al. 2016}
Figure~1 of Guo et al. (2016) shows a UV collapse in the quasar SDSS
J231742.60+000535.1 (hereafter J2317+0005, with redshift $z=0.32$) at
a very similar wavelength to that in the 2010 spectrum of J110057. The
collapse in J2317+0005 happens in 23 days \citep[Figure 2 of
][]{Guo2016}; this object was observed by SDSS on MJD 52177 (normal)
and then on MJD 52200. In the second epoch spectrum, there is a drop
of 1.2 mag in the $u$-band and 0.5mag in $g$-band. The $r,i,z$-bands
are all consistent with the earlier observation. J2317+0005 is
observed with SDSS $\sim$a year later and $(u,g,r,i,z)$ are all
consistent with the earlier MJD 52177 spectrum. XMM-Newton spectra
straddle this time period, from 2001 June 03 2001 November 28. Both
x-ray spectra are consistent with no neutral absorption in the
rest-frame. This implies the sightline is clear on both of those
dates.  \citet{Guo2016} also find that the IR does not significantly
change and that the broad lines are consistent with being constant
over time.

\citet{Guo2016} discuss two scenarios to explain this behaviour: {\it
(i)} an inner accretion disk change and {\it (ii)} an eclipse by an
optically thick cloud. \citet{Guo2016} note that in principle both
models could explain the observation. In the inner accretion disk
scenario, turning off the disk at $r < 60 r_{\rm g}$ would explain the
J2317+0005 spectra, though the detailed MCD fit is far from ideal, in much the same manner as for J11057. However,
\citet{Guo2016} find this explanation unconvincing since ``quasars are
not observed to flicker like this typically''.  The second scenario is
favoured based on the initial optical spectrum (23 days before the
$u$-band dip) and the 2001 November x-ray spectrum (45 days after the
$u$-band dip).  For the reasons given above for J110057, and in
particular with our infrared light curve data, we suggest $\approx$45
days is too short for an obscuration event, and hence favour a model
with a change in the inner accretion disk.


\bibliographystyle{mn2e}
\bibliography{/cos_pc19a_npr/LaTeX/tester_mnras}


\end{document}