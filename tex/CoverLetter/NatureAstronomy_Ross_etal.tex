%%%%%%%%%%%%%%%%%%%%%%%%%%%%%%%%%%%%%%%%%
% Professional Formal Letter
% LaTeX Template
% Version 1.0 (28/12/13)
%
% This template has been downloaded from:
% http://www.LaTeXTemplates.com
%
% Original author:
% Brian Moses (http://www.ms.uky.edu/~math/Resources/Templates/LaTeX/)
% with extensive modifications by Vel (vel@latextemplates.com)
%
% License:
% CC BY-NC-SA 3.0 (http://creativecommons.org/licenses/by-nc-sa/3.0/)
%
%%%%%%%%%%%%%%%%%%%%%%%%%%%%%%%%%%%%%%%%%

%----------------------------------------------------------------------------------------
%	PACKAGES AND OTHER DOCUMENT CONFIGURATIONS
%----------------------------------------------------------------------------------------

\documentclass[11pt,a4paper]{letter} % Specify the font size (10pt, 11pt and 12pt) and paper size (letterpaper, a4paper, etc)

\usepackage{graphicx} % Required for including pictures
\usepackage{microtype} % Improves typography
\usepackage{gfsdidot} % Use the GFS Didot font: http://www.tug.dk/FontCatalogue/gfsdidot/
\usepackage[T1]{fontenc} % Required for accented characters
%\usepackage[english,french]{babel}    
%\usepackage[english]{babel}    
\usepackage[margin=3.1cm]{geometry}
\usepackage[latin1]{inputenc}  
\usepackage{hyperref}


% Create a new command for the horizontal rule in the document which allows thickness specification
\makeatletter
\def\vhrulefill#1{\leavevmode\leaders\hrule\@height#1\hfill \kern\z@}
\makeatother

%----------------------------------------------------------------------------------------
%	DOCUMENT MARGINS
%----------------------------------------------------------------------------------------

\textwidth            6.5in    %     6.75in
\textheight           10.60in   %  10.75in %% 9.25
\oddsidemargin     -.25in
\evensidemargin    -.25in
\topmargin           -1.45in     % was -0.925
\longindentation     0.50\textwidth
\parindent               0.4in

%----------------------------------------------------------------------------------------
%	SENDER INFORMATION
%----------------------------------------------------------------------------------------

\def\Who{Nicholas P. Ross} % Your name
\def\What{, PhD} % Your title
\def\WhereOne{Institute for Astronomy} % Your department/institution
\def\WhereTwo{University of Edinburgh} % Your department/institution
\def\Address{Royal Observatory, Blackford Hill} % Your address
\def\Addressone{Royal Observatory} % Your address
\def\Addresstwo{Blackford Hill} % Your address
\def\CityZip{Edinburgh EH9 3HJ, U.K.} % Your city, zip code, country, etc
\def\Email{\href{mailto:npross@roe.ac.uk}{{\tt npross@roe.ac.uk}}} % Your email address
%\def\Email{npross@roe.ac.uk} % Your email address
\def\TEL{+44 (0)131-668 8351} % Your phone number
%\def\URL{\href{http://www.roe.ac.uk/~npross/Welcome.html}{http://www.roe.ac.uk/$\sim$npross/} % Your URL
\def\URL{\href{http://www.roe.ac.uk/~npross/}{www.roe.ac.uk/$\sim$npross}} % Your URL

%----------------------------------------------------------------------------------------
%	HEADER AND FROM ADDRESS STRUCTURE
%----------------------------------------------------------------------------------------

\address{
\includegraphics[width=1.2in]{University_of_Edinburgh_ceremonial_roundel.png} % Include the logo of your institution
%\includegraphics[width=1in]{avatar-logo-blueonwhite.png}
\hspace{5.1in} % Position of the institution logo, increase to move left, decrease to move right
\vskip -1.07in~\\ % Position of the text in relation to the institution logo, increase to move down, decrease to move up
\Large\hspace{1.5in}THE UNIVERSITY \hfill ~\\[0.05in] % First line of institution name, adjust hspace if your logo is wide
\hspace{1.5in}OF EDINBURGH \hfill \normalsize % Second line of institution name, adjust hspace if your logo is wide
\makebox[0ex][r]{\bf \Who \What }\hspace{0.08in} % Print your name and title with a little whitespace to the right
~\\[-0.11in] % Reduce the whitespace above the horizontal rule
\hspace{1.5in}\vhrulefill{1pt} \\ % Horizontal rule, adjust hspace if your logo is wide and \vhrulefill for the thickness of the rule
\hspace{\fill}\parbox[t]{2.85in}{ % Create a box for your details underneath the horizontal rule on the right
%\hspace{\fill}\parbox[t]{3.20in}{ % Create a box for your details underneath the horizontal rule on the right
\footnotesize % Use a smaller font size for the details
%\Who \\ \em % Your name, all text after this will be italicized
\WhereOne\\ % Your department
\WhereTwo\\ % Your department
%\Address\\
\Addressone \\ % Your address
\Addresstwo \\
\CityZip\\ % Your city and zip code
%\TEL\\ % Your phone number
\Email\\ % Your email address
%\URL % Your URL
}
\hspace{-1.4in} % Horizontal position of this block, increase to move left, decrease to move right
\vspace{-1in} % Move the letter content up for a more compact look
}

%----------------------------------------------------------------------------------------
%	TO ADDRESS STRUCTURE
%----------------------------------------------------------------------------------------

\def\opening#1{\thispagestyle{empty}
{\centering\fromaddress \vspace{0.6in} \\ % Print the header and from address here, add whitespace to move date down
\hspace*{\longindentation}\today\hspace*{\fill}\par} % Print today's date, remove \today to not display it
{\raggedright \toname \\ \toaddress \par} % Print the to name and address
%\vspace{0.4in} % White space after the to address
\vspace{0.25in} % White space after the to address
\noindent #1 % Print the opening line

% Uncomment the 4 lines below to print a footnote with custom text
%\def\thefootnote{}
%\def\footnoterule{\hrule}
%\footnotetext{\hspace*{\fill}{\footnotesize\em Footnote text}}
%\def\thefootnote{\arabic{footnote}}
}

%----------------------------------------------------------------------------------------
%	SIGNATURE STRUCTURE
%----------------------------------------------------------------------------------------

\signature{\Who \What} % The signature is a combination of your name and title

\long\def\closing#1{
\vspace{0.1in} % Some whitespace after the letter content and before the signature
\noindent % Stop paragraph indentation
\hspace*{\longindentation} % Move the signature right
\parbox{\indentedwidth}{\raggedright
#1 % Print the signature text
\vskip 0.35in % Whitespace between the signature text and your name
\fromsig}} % Print your name and title

%----------------------------------------------------------------------------------------

\begin{document}

%----------------------------------------------------------------------------------------
%	TO ADDRESS
%----------------------------------------------------------------------------------------

\begin{letter}
{
%Physics Department\\
%Lancaster University\\
%Lancaster\\
%LA1 4YB\\
%United Kingdom\\
}

%----------------------------------------------------------------------------------------
%	LETTER CONTENT
%----------------------------------------------------------------------------------------

\vspace{-10pt}
\opening{Dear Nature Astronomy,}
%\smallskip

%Authors should provide a cover letter that includes the affiliation and contact information for the corresponding author. Authors should briefly discuss the importance of the work and explain why it is considered appropriate for the diverse readership of Nature Astronomy. Any prior discussions with a Nature Astronomy editor about the work described in the manuscript should also be mentioned.

Accretion disks are ubiquitous astrophysical systems, appearing as
protoplanetary disks, Galactic X-ray Binary system and the active
galactic nuclei of quasars.  The accretion disks around supermassive
black holes are seen as the key mechanism in transporting both mass
and angular momentum.

The seminal paper by Shakura \& Sunyaev (1973, A\&A, 24, 337) laid out
the physical mechanisms and developed the mathematics for accretion
disks around stellar compact objects, i.e. black holes in binary
systems.  However, it was quickly realised that the physics could be
scaled up to supermassive black holes, which are known to power
quasars.

While both stellar X-ray binary systems and quasars were known to vary
in their luminosity output (indeed, this was one major motivation to
invoke a black hole plus accretion disk power source), the typical
timescales involved for quasars were months to years, making detailed
monitoring and observation hard.

However, with the recent advent of time-domain observational
astronomy, and in particular combing both photometric monitoring and
spectroscopic analyses, we are able to gain deep insights into
accretion disk physics extragalactic sources. 
%%
Indeed, these new data now deeply challenge the underlying
``$\alpha$-model'' paradigm initially developed by Shakura \&
Sunyaev. This was noted in the recent article by Lawrence (2018,
Nature Astronomy, Vol. 2, p.102) titled the ``Quasar viscosity
crisis''.

Here, for the first time, we marry the detailed observations and a
well-developed phenomenological model that explains the dramatic changes
recently observed in quasar accretion disks. This is appropriate for
publication in Nature Astronomy since it not only connects and tests
underlying accretion disk theory with novel observations, but also
makes strong predictions to how to interpret future time-domain
observations of the variable extragalactic universe. Furthermore, as
has been demonstrated by the recent Lawrence article, as well as the
Collection in last month's edition, AGN and quasars remain pivotal for
understanding galaxy evolution both observationally and theoretically.

We report on the quasar J1100-0053. The sister paper, Stern et
al. (2018), reports on a second quasar J1052+1519, also identified as
interesting due to its infrared properties, but showing changes in the
broad emission lines, rather than the continuum and associated
accretion disk. This paper was submitted to {\it The Astrophysical
Journal} earlier this month.

We are not aware of any arising conflicts of interest, and if you
require any further information, please do not hesitate to contact
me. Thank you very much for your consideration.




\closing{Sincerely,}

%----------------------------------------------------------------------------------------

\end{letter}
\end{document}
